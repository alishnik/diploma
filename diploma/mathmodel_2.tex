\subsection{Математическая модель №2}
\label{subsec:mathmodel_2}

Данная модель в общем случае рассматривает процесс передачи пакетов в случае известной очереди пакетов в начальный момент времени. Данное обстоятельство не использовалась в предыдущей модели, однако знание о текущей очереди может помочь подобрать наиболее подходящий период резервирования.

Идея этой модели основана на том, что в моменты времени, когда изначальная очередь не пуста, происходят однозначные переходы относительно размера следующей пачки. Когда очередь становится пустой, протекают всевозможные переходы по всем размерам следущей пачки.

Состояние системы описывается той же тройкой параметров, что и в \ref{subsec:mathmodel_1}. В каждый момент времени каждому состоянию $(h, m, n)$ будет соответствовать:
\begin{itemize}
\item его вероятность $\pi_{(h, m, n)}$;
\item среднее число потерянных пакетов по прибытию в данное состояние $\lost_{(h, m, n)}$;
\item среднее число \textit{обработанных} пакетов $\proc_{(h, m, n)}$, т.е. таких пакетов, которые по прибытию в данное состояние были либо успешно доставлены, либо отброшены. Это определение далее поможет в корректном вычислении $PLR$ в случае частично известного передаваемого потока.
\end{itemize}

Назовём \textit{фронтом состояний} системы в момент времени $t$ множество всех возможных состояний $(h(t), m(t), n(t))$, в которые система может попасть с ненулевой вероятностью. Будем рассматривать систему в моменты начал интервалов резервирования, в каждом из которых она имеет свой фронт $F(t)$.
%Будем считать, что если моменты резервирования и поступления пачки происходят в одно и то же время, то момент поступления пачки наступает раньше начала резервирования.

\subsubsection{Цепь Маркова в случае неизвестного передаваемого потока}
\label{subsec:mathmodel_2_unknown}

Будем считать известной матрицу переходных вероятностей $\{ p_{ij} \}_{i=1, j=1}^{i=M, j=M}$, а также распределение вероятностей размеров пачек $\{ p_i \}_{i=1}^{i=M}$.

Начнём рассматривать систему в момент наступления первого резервирования. Тогда начальный фронт будет состоять из состояний $(\xi, i, i)$, $i \in \{ 1, \ldots, M \}$, $\pi_{(\xi, i, i)} = p_i$, $\lost_{(\xi, i, i)} = \proc_{(\xi, i, i)} = 0$. Такая цепь Маркова ничем не отличается от цепи Маркова в \ref{subsec:mathmodel_1}, однако необходимо определить, сколько в среднем будет потеряно и обработано пакетов в каждом переходе. Таким образом, система  переходит из состояния $(h, m, n)$ в одно из следующих состояний:

\begin{tabular}{l l l l}
\label{mathmodel2_tab}
1)	&$(h + \tres, m, m)$,	&$\gamma = 1$,	&при $h < 0$,\\
	&$\Delta\lost = 0$,		&$\Delta\proc = 0$;\\
2)	&$(h + \tres - \tin, i, i)$,	&$\gamma = p_{ni}$,	&при $d - \tres< h$,\\
	&$\Delta\lost = m - 1 + \qdet$,	&$\Delta\proc = m$;\\
3)	&$(h + \tres, m, n)$,	&$\gamma = \qdet$,	&при $0 \leq h \leq d - \tres$,\\
	&$\Delta\lost = 0$,		&$\Delta\proc = 0$;\\
4)	&$(h - \tin + \tres, i, i)$,	&$\gamma = ( 1 - \qdet) p_{ni}$,	&при $m = 1$, $0 \leq h \leq d - \tres$,\\
	&$\Delta\lost = 0$,		&$\Delta\proc = 1$;\\
5)	&$(h + \tres, m - 1, n)$,	&$\gamma = ( 1 - \qdet)$,	&при $m > 1$, $0 \leq h \leq d - \tres$;\\
	&$\Delta\lost = 0$,		&$\Delta\proc = 1$;\\
%\caption{Таблица переходных вероятностей в цепи Маркова математической модели №2 в случае низвестного передаваемого потока}
\end{tabular} \newline
где $i\in\{1,\ldots,M\}$, $\gamma$ --- вероятность перехода, 	$\Delta\lost$, $\Delta\proc$ --- среднее число отброшенных и обработанных пакетов за переход соответственно.

%Тогда единственным отличием между переходами из состояний в фронте $F_i$  в состояния в фронте $F_{i+1}$ в цепи Маркова в случае неизвестного передаваемого потока и в случае заранее известного передаваемого потока является то, что переходы, где $n$ переходит в $n+1$ в случае заданного потока, в случае неизвестного потока $n$ переходит в $i$, а $m$ переходит в $i$ с вероятностью $p_{ni}$, $i \in \{ 1, \ldots, M \}$.

\subsubsection{Цепь Маркова в случае заранее известного передаваемого потока}
\label{subsec:mathmodel_2_know}

В данном случае рассмотрим передачу пакетов следующим образом. Будем считать, что размеры пачек входного потока  заранее известны, т.е. наперед известен размер $\flowsize$ потока $\flow$ в пачках и размер каждой из них $\flow_i, i = \overline{1, \flowsize}$ в пакетах. Состояние системы будет описываться той же тройкой параметров $(h,m,n)$, что и в \ref{subsec:mathmodel_2_unknown}, с той лишь поправкой, что $n$ равно номеру старшей пачки в потоке. Следовательно, $1 \leq n \leq \flowsize$.

В начальный момент времени система находится в состоянии $(\xi, B_1, 1)$. Переходы из фронта $F(t)$ в фронт $F(t+1)$ описываются \ref{mathmodel2_tab}, однако в ней надо учесть то, что поток заранее известен. Таким образом, получим следующие переходы из состояния $(h, m, n)$:

\begin{tabular}{l l l l}
\label{mathmodel2_tab_know}
1)	&$(h + \tres, m, n)$,	&$\gamma = 1$,	&при $h < 0$,\\
	&$\Delta\lost = 0$,		&$\Delta\proc = 0$;\\
2)	&$(h + \tres - \tin, \flow_{n+1}, n + 1)$,		&$\gamma = 1$,	&при $d - \tres< h$, $n < \flowsize$\\
	&$\Delta\lost = m - 1 + \qdet$,	&$\Delta\proc = m$;\\
3)	&$(h + \tres, m, n)$,	&$\gamma = \qdet$,	&при $0 \leq h \leq d - \tres$,\\
	&$\Delta\lost = 0$,		&$\Delta\proc = 0$;\\
4)	&$(h - \tin + \tres, \flow_{n+1}, n+1)$,		&$\gamma = ( 1 - \qdet)$,	&при $m = 1$, $0 \leq h \leq d - \tres$, $n < \flowsize$\\
	&$\Delta\lost = 0$,		&$\Delta\proc = 1$;\\
5)	&$(h + \tres, m - 1, n)$,	&$\gamma = ( 1 - \qdet)$,	&при $m > 1$, $0 \leq h \leq d - \tres$;\\
	&$\Delta\lost = 0$,		&$\Delta\proc = 1$;\\
%\caption{Таблица переходных вероятностей в цепи Маркова математической модели №2 в случае низвестного передаваемого потока}
\end{tabular} \newline
где $i\in\{1,\ldots,M\}$, $\gamma$ --- вероятность перехода, 	$\Delta\lost$, $\Delta\proc$ --- среднее число отброшенных и обработанных пакетов за переход соответственно. В дополнение, все переходы имеют место при условии, что новое состояние $(h,m,n)$ имеет $n \leq \flowsize$.

%Рассмотрим возможные переходы из состояния $(h, m, n)$ в фронте $F_i$ $i$-ого резервирования в состояния в фронте $F_{i + 1}$ $(i+1)$-ого резервирования, а также прирост среднего числа потерянных пакетов за эти переходы. Также для каждого состояния будем находить прирост количества за переход \textit{обработанных} пакетов, т.е. тех пакетов, которые либо переданы, либо отброшены.

%\paragraph{1.} $h < 0$, т.~е. очередь пуста.
%
%\begin{itemize}
%\item Если $h + \Tres \geq 0$, то к моменту времени $t + 1$ в очередь поступит очередная пачка пакетов. Размер пачки заранее определён, поэтому с вероятностью $1$ система окажется в состоянии $(h + \Tres, B_{i+1}, i+1)$.
%
%\item Если же $h + \Tres < 0$, то к моменту времени $t + 1$ в очередь не поступит ни одного пакета, т.~е. в этом случае система с вероятностью $1$ перейдет в состояние $(h + \Tres, 0, i)$. 
%\end{itemize}
%
%В обоих случаях среднее число потерянных пакетов и число обработанных пакетов за переход равно нулю. 
%%%%W%%%%%%%%%%%%
%\paragraph{2.} Пусть $D \geq h(t) \geq 0$ и $m(t) = 1$, т.~е. в головной пачке находится единственный пакет.
%
%С вероятностью $1 - \qdet$ происходит успешная передача пакета.
%
%\begin{itemize}
%\item Если при этом $h - \Tin + \Tres < 0$, то к моменту времени $t + 1$ на станцию не поступит ни одной пачки данных и система окажется в состоянии $(h - \Tin + \Tres, 0, n)$ с вероятностью $1 - \qdet$.
%\item Если же $h - \Tin + \Tres \geqslant 0$, то в момент времени $t + 1$ очередь будет не пуста, и состояние системы будет определяться числом пакетов в головной пачке с номером  $n + 1$, тогда система перейдет в состояние $(h - \Tin + \Tres, B_{n+1}, n+1)$ с вероятностью $1 - \qdet$.
%\end{itemize}
%
%В обоих этих случаях среднее число потерянных пакетов за переход равно нулю. Число обработанных пакетов равно 1.
%
%С вероятностью $\qdet$ происходит неуспешная передача пакета.
%
%\begin{itemize}
%\item Если $h + \Tres > D$, то к моменту времени $t + 1$ время жизни данного пакета  превысит допустимое значение $D$, и этот пакет будет отброшен. Cистема c вероятностью $\qdet$ перейдет в состояние $(h - \Tin + \Tres, B_{n+1}, n + 1)$. Среднее число потерянных пакетов в этом переходе равно $\qdet$. Число обработанных пакетов равно 1.
%\item Если $h + \Tres \leq D$, то к моменту времени $t + 1$ пакет не устареет, и будет предпринята следующая попытка передачи. Следовательно, система перейдет в состояние $(h + \Tres, 1, n)$ с вероятностью $\qdet$. Среднее число потерянных пакетов в этом переходе равно нулю. Число обработанных пакетов равно нулю.
%\end{itemize}
%
%\paragraph{3.} Наконец, рассмотрим случай, когда $h(t) \geq 0$, $m(t) > 1$. 
%
%\begin{itemize}
%\item Если $h + \Tres > D$, то к моменту времени $t + 1$ время ожидания этих пакетов превысит допустимое значение $D$, и все пакеты пачки будут отброшены. Система с вероятностью $1$ перейдет в состояние $(h - \Tin + \Tres, B_{n+1}, n+1)$. Здесь среднее число потерянных пакетов за переход равно $m - 1 + \qdet$. Число обработанных пакетов равно $m$.
%\item Если $h + \Tres \leq D$, то система с вероятностью $1 - \qdet$ перейдет в состояние $(h + \Tres, m - 1, n)$, где число обработанных пакетов равно 1, а с вероятностью $\qdet$ перейдёт в состояние $(h + \Tres, m, n)$, где число обработанных пакетов равно нулю.
%\end{itemize}
%
%Здесь в обоих переходах среднее число потерянных пакетов равно нулю.
%
%После описания всевозможных переходов из фронта $F_i$ в фронт $F_{i+1}$ необходимо отнормировать все вероятности (в следующий фронт цепь попадает с вероятностью 1).
%
%\paragraph{Подводя итог,} получаем, что при выполнении соответствующих условий система  переходит из состояния $(h, m, n)$ в одно из следующих состояний:
%
%\begin{tabular}{l l l l l}
%1)	&$(h + \Tres, 0, n)$, 			&$\gamma = 1$, &$\Delta N_{lost} = 0$,	&при $m = 0$, $h < -\Tres$; \\
%2)  &$(h + \Tres, B_{n+1}, n+1)$, 			&$\gamma = 1$, &$\Delta N_{lost} = 0$, &при $m = 0$, $-\Tres \leqslant h < 0$; \\
%3)  &$(h + \Tres - \Tin, B_{n+1}, n+1)$, 	&$\gamma = 1 - \qdet$,  &$\Delta N_{lost} = 0$, &при $m = 1$, $\Tin - \Tres \leqslant h \leqslant D$; \\
%4)  &$(h + \Tres - \Tin, 0, n)$, 	&$\gamma = 1 - \qdet$,  &$\Delta N_{lost} = 0$, &при $m = 1$, $0 \leqslant h < \Tin - \Tres$; \\
%5)  &$(h + \Tres - \Tin, B_{n+1}, n + 1)$, 	&$\gamma = \qdet$,  &$\Delta N_{lost} = \qdet$, &при $m = 1$, $d - \Tres < h \leqslant D$; \\
%6)  &$(h + \Tres, m, n)$, 			&$\gamma = \qdet$,  &$\Delta N_{lost} = 0$, &при $m > 0$, $0 \leqslant h \leqslant D - \Tres$; \\
%7)  &$(h + \Tres - \Tin, B_{n+1}, n+1)$,	&$\gamma = 1$,  &$\Delta N_{lost} = m - 1 + \qdet$, &при $m > 1$, $d - \Tres < h \leqslant D$; \\
%8)  &$(h + \Tres, m - 1, n)$, &$\gamma = 1 - \qdet$,  &$\Delta N_{lost} = 0$, &при $m > 1$, $0 \leqslant h \leqslant D - \Tres$;
%\end{tabular} \newline
%где $\gamma$ --- вероятность перехода,  $\Delta N_{lost}$ --- среднее число потерянных пакетов за переход.

\subsubsection{Цепь Маркова в случае частично известного передаваемого потока}
\label{subsec:mathmodel_2_mixed}

Можно объединить две вышеописанные цепи Маркова, объединив в том числе начальные условия: пусть имеется известный входной поток $\flow$ размера $\flowsize$, распределение вероятностей размеров пачек $\{ p_i \}_{i=1}^{i=M}$, матрица переходных вероятностей $\{ p_{ij} \}_{i=1, j=1}^{i=M, j=M}$.
%, введя новый параметр $r$, являющийся индикатором об известности входного потока: если $r$ = 0, то поток в данном состоянии известен, иначе $r$ = 1.

Состояние описывается тройкой параметров $(h, m, n)$.
%$r$ --- вышеописанный индикатор об известности входного потока,
$h$ --- возраст головной пачки, $m$ --- количество оставшихся пакетов в головной пачке, а $n$ --- номер головной пачки, если $n \leq \flowsize$, в противном случае это начальный размер головной пачки.

В данной модели начальный фронт $F(0)$ содержит только одно начальное состояние $(\xi, \flow_1, 1)$. Далее, до тех пор, пока $n \leq l$, все переходы описываются так же, как и в \ref{subsec:mathmodel_2_know}. Но в случаях, когда по логике модели \ref{subsec:mathmodel_2_know}, значение $n$ должно быть равным $\flowsize + 1$, вместо этого состояние $(h, m, \flowsize)$ перейдёт в состояние $(h + \Tres - \Tin, i, i)$ с вероятностью $p_{\flow_\flowsize i}$, $i \in \{ 1, \ldots, M \}$. При этом среднее число потерянных пакетов за переход будет равно 0, если $m = 0$, или $m - 1 + \qdet$, если $m \geq 1$.
Далее происходят переходы, описанные в \ref{subsec:mathmodel_2_unknown}.

\subsubsection{Определение PLR}

Описав всевозможные переходы из состояний фронта $F(t)$ в состояния фронта $F(t+1)$, необходимо вычислить $\pi_{(h, m, n)}$, $\lost_{(h, m, n)}$, $\proc_{(h, m, n)}$ для каждого $(h, m, n) \in F(t+1)$.

Сначала вычисляются $\pi_{(h, m, n)}$:

\begin{equation*}
\pi_{(h, m, n)} = \sum\limits_{(h', m', n') \in F(t)} \pi_{(h', m', n')} p\left( (h', m', n') \to (h, m, n) \right),
\end{equation*}
где $p\left( (h', m', n') \to (h, m, n) \right)$ --- переходная вероятность из состояния $(h', m', n')$ в состояние $(h, m, n)$, каждая из которых описана в \ref{mathmodel2_tab}. После вычисления всех вероятностей их необходимо отнормировать (система перейдёт в одно из состояний следующего фронта с вероятностью 1).

Далее вычисляются $\lost_{(h, m, n)}$ и $\proc_{(h, m, n)}$:

\begin{equation*}
\lost_{(h, m, n)} = \sum\limits_{(h', m', n') \in F(t)} \pi_{(h', m', n')} p\left( (h', m', n') \to (h, m, n) \right) \cdot (\Delta\lost_{(h', m', n') \to (h, m, n)} + \lost_{(h', m', n')});
\end{equation*}
\begin{equation*}
\proc_{(h, m, n)} = \sum\limits_{(h', m', n') \in F(t)} \pi_{(h', m', n')} p\left( (h', m', n') \to (h, m, n) \right) \cdot (\Delta\proc_{(h', m', n') \to (h, m, n)} + \proc_{(h', m', n')}).
\end{equation*}

Будем рассматривать передачу только в течение временного промежутка $W$. Следовательно, в цепи Маркова нужно совершить переходы до фронта в момент времени:
$$
t_{stop} = \floor*{\frac{W - \xi}{\Tres} + 1},
$$
который есть номер первого резервирования при условии, что это резервирование выходит за пределы окна $W$.

Зная всевозможные состояния в последнем фронте $F(t_{stop})$ и соответствующее каждому из них среднее число отброшенных пакетов $\lost_{(h, m, n)}$ и число обработанных пакетов $\proc_{( h, m, n)}$, можно вычислить среднее число отброшенных пакетов к моменту наступления фронта $F(t_{stop})$:
$$
PLR = \sum\limits_{(h,m,n) \in F(t_{stop})} \pi_{(h,m,n)} \cdot \frac{\lost_{(h, m, n)}}{\proc_{(h, m, n)}}.
$$

С помощью вышеописанной модели можно найти параметры резервирования таким образом, чтобы удовлетворять QoS-требованиям.