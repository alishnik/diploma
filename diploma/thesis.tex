\documentclass[10pt, bachelor,subf,href,colorlinks=true
%,fixint=false
%,times
]{disser}

\usepackage[
  a4paper, mag=1000, includefoot,
  left=2.5cm, right=1.5cm, top=2cm, bottom=2cm, headsep=1cm, footskip=1cm
]{geometry}
\usepackage{mathtext}
\usepackage[T2A]{fontenc}
\usepackage[utf8]{inputenc}
%\usepackage{ucs}
\usepackage[english,russian]{babel}
\usepackage{graphicx}
\usepackage{cmap}
\usepackage{algorithm}
\usepackage{algpseudocode}
\usepackage{mathtools}
\DeclarePairedDelimiter{\ceil}{\lceil}{\rceil}
\DeclarePairedDelimiter{\floor}{\lfloor}{\rfloor}
%\usepackage{subfig}
%\usepackage{tikz}
%\usepackage{float}
%\usepackage{caption}
\usepackage[font=small]{subcaption}
%AMS TEX значки и пр.
\usepackage{amssymb, amsthm, amsmath}
%\usepackage[noend]{algorithmic}
%\usepackage{array}
\usepackage{tabularx}
\usepackage{multicol, multirow}
\usepackage{ragged2e}
\ifpdf\usepackage{epstopdf}\fi

% Номера страниц сверху и по центру
%\def\headfont{\small}
%\pagestyle{headcenter}
%\chapterpagestyle{empty}

% Использовать полужирное начертание для векторов
\let\vec=\mathbf

% Включать подсекции в оглавление
\setcounter{tocdepth}{2}

\graphicspath{{pics/}}

\numberwithin{equation}{section}
\begin{document}
\newcommand{\hmax}{\ensuremath{h^{max}}}
\newcommand{\hmin}{\ensuremath{h^{min}}}

\newcommand{\hini}[1]{h^{(i)}_{#1}}
\newcommand{\hfin}[1]{h^{(f)}_{#1}}

\newcommand{\Tin}{\ensuremath{T^{in}}}
\newcommand{\Tout}{\ensuremath{T^{out}}}
\newcommand{\Tres}{\ensuremath{T^{res}}}
\newcommand{\tin}{\ensuremath{t^{in}}}
\newcommand{\tout}{\ensuremath{t^{out}}}
\newcommand{\tres}{\ensuremath{t^{res}}}
\newcommand{\PER}{\ensuremath{q}}
\newcommand{\Iin}{\ensuremath{I^{in}}}
\newcommand{\Idis}{\ensuremath{I^{dis}}}
\newcommand{\Iout}{\ensuremath{I^{out}}}

\newcommand{\Qmax}{\ensuremath{Q^{max}}}

\newcommand{\DQoS}{\ensuremath{D^{QoS}}}
\newcommand{\PLRQoS}{\ensuremath{PLR^{QoS}}}
\newcommand{\qdet}{\ensuremath{q^{det}}}
\newcommand{\qran}{\ensuremath{q^{ran}}}

\renewcommand{\le}{\leqslant}
\renewcommand{\ge}{\geqslant}

\newcommand{\pin}{\ensuremath{p^{in}}}
\newcommand{\Pin}{\ensuremath{P^{in}}}
\newcommand{\pout}{\ensuremath{p^{out}}}
\newcommand{\Pout}{\ensuremath{P^{out}}}



\institution{Московский физико-технический институт \\ (государственный университет) \\ Факультет радиотехники и кибернетики \\ Кафедра проблем передачи и обработки информации}

% Имя лица, допускающего к защите (зав. кафедрой)
\apname{такой-то}

\title{Выпускная квалификационная работа\\БАКАЛАВРА}

\topic{Передача мультимедийных данных в сетях Wi-Fi Mesh с помощью механизма детерминированного доступа}

% Автор
\author       {Бабаев А.А.} % ФИО
\group        {311} % Группа
\coursenum    {010900} % Номер направления
\course       {Прикладные математика и физика}
%\masterprognum{010674} % Номер магистерской программы
%\masterprog   {Телекоммуникационные сети и системы}

% Научный руководитель
\sa      {Хоров Е.М.}
\sastatus{к.~т.~н.}

% Рецензент
%\rev      {Якимов М.Ю.}
%\revstatus{к.~т.~н.}

% Консультант
%\conspec{вопросам\\охраны труда}
%\con{ФИО консультанта}
%\constatus{к.~т.~н., доц.}

% Город и год
\city{Москва}
\date{\number\year}

\maketitle
\newpage

\tableofcontents

\intro
%\textbf{Актуальность задачи.} С ростом популярности сервисов реального времени растут как объемы генерируемых ими мультимедийных данных, так и требования пользователей к качеству обслуживания таких данных, QoS-требования (Quality of Service). QoS-требования данных реального времени включают в себя не только минимальную пропускную способность, но и ограничения на задержку передачи и долю потерянных пакетов PLR (Packet Loss Ratio). Выполнение QoS-требований при передаче мультимедийных данных с минимальными затратами канальных ресурсов является одной из наиболее актуальных задач в области беспроводных коммуникаций.

\textbf{Актуальность задачи.} В течение десятилетий растёт популярность сервисов реального времени, а вместе с ней неизбежно растут размеры генерируемых ими мультимедийных данных, а также ужесточаются требования пользователей к качеству обслуживания таких данных, QoS-требования (Quality of Service). QoS-требования данных реального времени включают в себя минимальную пропускную способность. Более того, к QoS-требованиям относятся ограничения на задержку передачи и долю потерянных пакетов PLR (Packet Loss Ratio). Передача мультимедийных данных с минимальными затратами канальных ресурсов и выполнением QoS-требований является одной из наиболее актуальных задач в области беспроводных коммуникаций.

В сетях Wi-Fi возможность выполнения QoS-требований зависит от используемых механизмов доступа к каналу. При использовании механизмов \textit{детерминированного} доступа станции передают в заранее зарезервированных ими интервалах времени. С одной стороны, передачи в зарезервированных интервалах защищены от коллизий, поскольку станции не ведут передачу в чужих интервалах. С другой стороны, изменение зарезервированных интервалов занимает существенное время, затрачиваемое на согласование новых параметров зарезервированных интервалов с соседними станциями. Такие задержки могут быть особенно критичны в случае передачи мультимедийных потоков реального времени, кратковременные всплески интенсивности которых вынуждают либо изменять параметры зарезервированных интервалов, либо резервировать канальные ресурсы с избытком.

При использовании механизмов \textit{случайного} доступа станции выбирают случайные моменты времени для передачи данных. Это обеспечивает б\'{о}льшую гибкость по сравнению с детерминированным доступом, поскольку при их использовании для передачи данных можно попытаться получить доступ к каналу после того, как он оказался свободным в течение некоторого промежутка времени, а не ждать ближайшего зарезервированного интервала. По этой причине механизмы случайного доступа эффективны для поглощения всплесков интенсивности передаваемых потоков. В то же время таким механизмам свойственны частые коллизии (особенно в плотных сетях) и случайные задержки при попытках получить доступ к каналу. В результате механизмы случайного доступа не могут гарантировать выполнение QoS-требований при передаче данных реального времени. 

Механизм детерминированного доступа может быть улучшен за счет гибкости случайного доступа,  если использовать эти два вида доступа совместно в составе некоторого механизма \textit{комбинированного} доступа: основная часть потока может быть передана внутри зарезервированных интервалов, в то время как для поглощения всплесков интенсивности передаваемого потока дополнительно задействуется случайный доступ. Комбинированный доступ обещает значительные выгоды как с точки зрения возможности выполнения QoS-требований (за счет передачи внутри защищенных зарезервированных интервалов), так и с точки зрения объема используемого канального времени (не нужно резервировать время с расчетом на наибольшую возможную интенсивность входного потока, а только для передачи его постоянной составляющей). Основной вопрос заключается в том, как именно нужно использовать механизмы комбинированного доступа при передаче данных реального времени.
 
\textbf{Задачей данной работы} является анализ использования механизма комбинированного доступа для передачи мультимедийных данных реального времени и разработка метода выбора таких его параметров, при которых выполнены QoS-требования к передаче потока, а потребление канальных ресурсов минимально. Для решения данной задачи в работе разработана математическая модель передачи мультимедийного потока с использованием  механизма комбинированного доступа. Данная модель позволяет сравнить механизм комбинированного доступа с механизмом детерминированного доступа в плане потребления канальных ресурсов с необходимыми для выполнения QoS-требованиями к передаче потока. При построении математической модели были использованы такие \textbf{методы исследования}, как теория вероятностей и комбинаторный анализ.    

\textbf{Научная новизна} данной работы заключается в том, что в ней разработана новая математическая модель процесса передачи реального мультимедийного потока с использованием механизма комбинированного доступа с учетом QoS-требований и возможных ошибок при передаче пакетов.

\textbf{Положение, выносимое на защиту.} В данной работе была разработана новая математическая модель процесса передачи реального мультимедийного потока с использованием механизма комбинированного доступа. Разработанная модель позволяет найти такие параметры комбинированного доступа, при которых выполнены QoS-требования передаваемого потока, а потребление канальных ресурсов минимально.

\textbf{Данная работа построена следующим образом.} В разделе~\ref{sec:problem_title} приводится краткое описание рассматриваемых в работе механизмов доступа, обзор существующих работ и  формальная постановка задачи. Раздел~\ref{sec:math_model} посвящен построению математической модели рассматриваемого процесса передачи. В разделе~\ref{sec:numerical} разработанная модель применяется для выбора параметров передачи, демонстрируется выгода от использования предложенного механизма доступа по сравнению с использованием механизма детерминированного доступа. Наконец, итоги работы приводятся в заключении.
\newpage
\section{Постановка задачи}
\label{sec:problem_title}

\subsection{Методы доступа к среде}
\label{sec:methods}

Дополнение IEEE 802.11e~\cite{802.11e} к стандарту Wi-Fi, направленное на внедрение поддержки QoS-требований, определило два новых механизма доступа: EDCA (Enhanced Distributed Channel Access) и HCCA (Hybrid coordination function Controlled Channel Access). Первый из них является механизмом приоритезированного случайного доступа к каналу, а второй --- механизмом детерминированного доступа.

С ростом плотности Wi-Fi-сетей HCCA и EDCA оказались не способны обеспечить надежную передачу данных, так как не обладают возможностями по работе в условиях жесткой интерференции. В частности, в случае большого числа точек доступа, работающих в одной области пространства, многие точки доступа выбирают один и тот же частотный канал. В результате EDCA страдает от множественных коллизий и эффекта скрытых станций. Ситуация с HCCA лишь немногим лучше за счет того, что HCCA исключает интерференцию со станциями своей сети. Поэтому возникает необходимость в разработке новых механизмов доступа, рассчитаных на работу в плотных сетях Wi-Fi.

Например, дополнение IEEE 802.11aa определяет механизм HCCA TXOP Negotiation, который позволяет точке доступа зарезервировать последовательность периодических интервалов времени, в течение которых она опрашивает свои станции, а соседние точки доступа (вместе со своими станциями) хранят молчание.  Данный подход существенно уменьшает интерференцию с соседними сетями, повышая надежность передачи. Стоит отметить, что интервалы времени резервируются не по одному, а целыми последовательностями: точка доступа резервирует \textit{последовательность периодических интервалов времени равной длительности}, в дальнейшем называемую \textit{периодическим резервированием}. Благодаря периодичности, все резервирование может быть описано с помощью всего лишь трех величин: \textit{периода} зерезервированных интервалов, их \textit{длительности} и \textit{момента начала первого интервала} (рис.~\ref{fig:periodic_reservation}).

\begin{figure}[h]
\includegraphics[width=\linewidth]{ReservedIntervals.pdf}
\caption{\label{fig:periodic_reservation} Периодическое резервирование}
\end{figure} 

Кроме дополнения IEEE 802.11aa периодические резервирования также присутствуют и в дополнении IEEE 802.11s (описывающим сети Wi-Fi Mesh). Это дополнение определяет механизм детерминированного доступа MCCA (Mesh coordination function Controlled Channel Access). MCCA позволяет любой станции в сети Wi-Fi Mesh установить периодическое резервирование, чтобы защитить собственные передачи от интерференции с соседними станциями. 

При использовании механизма комбинированного доступа для передачи мультимедийных данных реального времени с учетом QoS-требований, необходимо ответить на вопрос о том, как должны быть размещены зарезервированные интервалы времени, и когда станция должна использовать механизм случайного доступа. Первая попытка ответить на этот вопрос была сделана в дополнении  IEEE 802.11e, которое описывает механизм доступа, называемый HEMM (HCCA-EDCA Mixed Mode). Однако, в описании HEMM не указан способ выбора параметров, и только несколько работ  \cite{kuan2007utilization, lai2009adaptation, Ng2012, ruscelli2012enhancement} рассматривают использование этого механизма для передачи данных с QoS-требованиями.
\subsection{Обзор существующих работ}
\label{sec:related_papers}

Часть работ, рассматривающих совместное использование случайного и детерминированного доступа, посвящены тому, как именно распределять ресурсы между этими механизмами. Например,~\cite{siris2006resource} показывает, как разделить ресурсы между EDCA и HCCA для достижения максимальной пропускной способности в сетях IEEE 802.11e. Сосуществование двух механизмов доступа также рассмотрено в~\cite{Krasilov2013} в рамках использования EDCA и MCCA в сетях Wi-Fi Mesh. \cite{Krasilov2013}~показывает, как количество зарезервированного времени влияет на пропускную способность EDCA. Однако в данных работах не рассмотрены ни совместное использование двух механизмов доступа для передачи одного потока данных, ни вопрос выполнения QoS-требований. 

%A number of papers devoted to both random and deterministic access study how to share resources between these methods.
%For example,~\cite{siris2006resource} gives recommendation on how to share resources between EDCA and HCCA to achieve the maximum overall throughput in IEEE 802.11e networks. The coexistence of the methods is also studied in~\cite{Krasilov2013} in terms of EDCA and MCCA in Wi-Fi Mesh networks. \cite{Krasilov2013}~shows how the amount of the reserved channel time influences the throughput of EDCA. But no QoS provisioning or joint usage is considered in both~\cite{siris2006resource} and~\cite{Krasilov2013}.

Что касается комбинированного использования двух механизмов доступа, то на эту тему существует всего лишь несколько работ, и все они в основном рассматривают HEMM. К примеру, \cite{kuan2007utilization} рассматривает ситуацию, в которой несколько станций передают в режиме насыщения, используя как EDCA, так и HCCA. Показано, что чем больше данных передается с помощью HCCA, тем больше доля времени, потраченного на успешные передачи. \cite{Ng2012}~предлагает использовать марковский процесс принятия решений для координирования HCCA и EDCA в сетях  IEEE 802.11e. Однако \cite{Ng2012}~не учитывает QoS-требования и ставит задачей лишь увеличение доли времени, потраченного на успешные передачи. Кроме того, ни  \cite{kuan2007utilization} ни \cite{Ng2012} не изучают применение HEMM для передачи потока с QoS-требованиями.

%As for the joint access, only a few studies exist so far and they mainly consider HCCA-EDCA Mixed Mode (HEMM). For example, \cite{kuan2007utilization} considers a scenario where a set of STAs transmit saturated traffic using both EDCA and HCCA. It is shown that the more data is transmitted with HCCA the higher is the channel utilization, that is, the percentage of time spend for successful transmissions. \cite{Ng2012}~proposes to use a Markov decision process for coordinating HCCA and EDCA in IEEE 802.11e networks. However, \cite{Ng2012}~considers no QoS requirements and aims only to improve the channel utilization. 
%In this paper, we propose a joint access method. But neither \cite{kuan2007utilization} nor \cite{Ng2012} study the ability of HEMM to transmit QoS-sensitive data. 

Влияние HEMM на выполнение QoS-требований рассмотрено в~\cite{ruscelli2012enhancement}, где представлены описание и анализ EDCA, HCCA и HEMM. \cite{ruscelli2012enhancement} предлагает улучшенный алгоритм управления доступом, который применяется для HEMM и может быть использован для улучшения показателей практически любого HCCA планировщика, что подтверждено имитационным моделированием. При этом не было разработано математической модели, которая могла бы быть использована для настройки параметров комбинированного доступа для поддержки QoS-требований.  

%The influence of HEMM on QoS provisioning is studied in~\cite{ruscelli2012enhancement} where a good description/analysis of EDCA, HCCA as well as HEMM can be found. \cite{ruscelli2012enhancement} proposes an enhanced admission control algorithm which accounts for HEMM and can be used to improve performance of almost any HCCA scheduler as shown by simulation. However, no analytical model, which can be used to study how joint access is efficient for supporting QoS, has been developed so far. Realizing this gap, we try to fill it in the current paper.

\subsection{Формальная постановка задачи}
\label{sec:problem_statement}


Рассмотрим передачу мультимедийного потока реального времени между двумя соседними станциями. Считается, что пакеты потока поступают в очередь пачками случайного размера (размер пачки принимает значение $i$ при учёте структуры уже известного на данный момент видеопотока $B(t)_det$ с вероятностью $\pi$, $i \in \{1, \ldots, MAXSIZE\}$) строго периодически с периодом $\Tin$, что соответствует передаче видеопотока с использованием протокола RTP. Далее такой поток будем называть периодическим, неординарным и коррелированным, то есть размер каждой следующей пачки зависит от его предыстории. QoS-требования представлены а) ограничением на время доставки пакета $\DQoS$ и б) ограничением на долю потерянных пакетов $\PLRQoS$. Если время нахождения пакета в очереди превышает $\DQoS$, то пакет отбрасывается, внося вклад в PLR.

Отправитель передает поток с использованием механизма комбинированного доступа, который работает следующим образом.
Отправитель устанавливает периодическое резервирование  с периодом $\Tres \leq \Tin$ и длительностью каждого зарезервированного интервала, рассчитанной ровно на одну попытку передачи пакета.
Отправитель всегда передает первый пакет из очереди.
%Для каждого пакета отправитель поддерживает счетчик передач: пакет может быть передан не более чем $\nretries$ раз.
%Спустя $\nretries$ попыток передачи пакет покидает очередь.
Также пакет покидает очередь, если был успешно передан, либо если его время нахождения в очереди превысило $\DQoS$.
Отправитель передает пакеты, используя либо детерминированный доступ (передавая пакеты в зарезервированных интервалах), либо случайный доступ. 
Передача с использованием случайного доступа ведется в промежутках между зарезервированными интервалами, т.е. в интервалах случайного доступа.
При этом пакет передается с использованием случайного доступа только тогда, когда он устаревает к началу следующего зарезерезвированного интервала.
В интервале случайного доступа можно совершить ограниченное количество попыток передач.
Время между двумя попытками передач в интервале случайного доступа считается  экспоненциально-распределенной случайной величиной с параметром $\lambda$. 

Вероятность неуспешной передачи пакета внутри зарезервированного интервала равна $\qdet$, а в течение интервала случайного доступа --- $\qran \geq \qdet$. 

При таком использовании механизма комбинированного доступа основной вопрос состоит в том, как выбрать параметр $\Tres$, чтобы выполнить QoS-требования, используя при этом как можно меньше канального времени. Для ответа на этот вопрос была разработана математическая модель рассматриваемого процесса передачи. Эта модель позволяет найти PLR как функцию от $\Tres$ и по зависимости $PLR(\Tres)$ найти искомый параметр.

Также был разработан алгоритм нахождения максимального значения $\Tres$, при котором выполняются QoS-требования, в динамическом режиме, учитывающий состояние очереди в моменты принятия решения, а также распределение размеров пришедших к этому моменту пачек.

%В течении зарезервированного интервала передача пакета происходит неудачно с вероятностью $\perm$. Если для передачи используется метод случайного доступа (пакет передается за пределами зарезервированного интервала), то передача происходит неудачно с вероятностью $\pere$. 

%Мы также предполагаем, что в какой бы момент до следующего зарезервированного интервала пакет ни устарел, времени на совершение всех необходимых передач с использованием метода случайного доступа будет достаточно. 

%Наконец, \textit{пакет отбрасывается, если он не был передан успешно, только если его счетчик количества попыток передач достигает значения $\nretries$}.         







\chapter{Математическая модель одиночной передачи пакетов}
\label{chapter:model_noBlockAck}

При анализе новой математической модели было произведено её сравнение с уже существующими моделями. Проведём их детальное описание.

\section{Аналитическая модель передачи}
\label{analyticalModel1}
\subsection{Цепь Маркова}

Представим $\Tin/\Tres$ в виде несократимой дроби $\tin/\tres$, где $\tin, \tres\in\mathbb{N}$. Назовем слотом интервал времени длины $$\tau = \frac{\Tin}{\tin} \equiv \frac{\Tres}{\tres}.$$ Разобьем непрерывную временную шкалу на слоты таким образом, чтобы начало каждого MCCAOP совпадало с началом некоторого слота -- см.~рис~\ref{fig:time}.

Процесс передачи пакетов с помощью механизма MCCA может быть описан цепью Маркова с дискретным временем, единица которого равна периоду резервирований.

%\begin{figure}[t]
%\centering{\includegraphics[width=0.7\linewidth]{pics/time1}}
%	\caption{\label{fig:time}Слотированное время}
%\end{figure}


В каждый момент времени $t$ состояние системы будем описывать парой целых чисел $(h(t), m(t), n(t))$. Если $h(t) \geqslant 0$, то очередь не пуста, и $h(t)$ равно числу полных слотов, которые головная (самая старшая) пачка пакетов провела в очереди, $m(t)$ равно числу пакетов в этой пачке, а $n(t)$ равно начальному числу пакетов в головной пачке. Если $h(t) < 0$, то очередь пуста, и $|h(t)|$ равно времени до прибытия новой пачки пакетов, выраженному в слотах и округленному вниз; при этом $m=0$. Таким образом, состояние системы в каждый моменты времени $t$ характеризуется текущим числом пакетов в головной пачке, начальным числом пакетов  в головной пачке и временем, в течение которого пакеты этой пачки ожидают передачи. Используемые обозначения состояния системы позволяют определить количество пачек пакетов в очереди:  $\lfloor h(t)/\tin \rfloor + 1$, но не позволяют определить длины всех пачек, кроме головной. После того, как все пакеты головной пачки были переданы или отброшены, определяется число пакетов в пачке, ставшей головной.
  
Минимальное значение $h(t)$ равно $\tres-\tin$. Оно достигается в момент времени $t$, если пачка из одного пакета пребывает в пустую очередь непосредственно перед моментом $t - 1$, и единственный пакет пачки успешно передается с первой попытки.   

Теперь найдем максимальное возможное значение $h(t)$. Для этого обозначим через $\xi$ время между поступлением пачки пакетов в очередь и началом следующего слота, $0 \leqslant \xi < \tau$. В силу того, что время $\Tin$ равно целому числу $\tin$ слотов, значение величины $\xi$ одинаково для всех пачек. Таким образом, к моменту времени $t$ время ожидания в очереди пакетов головной пачки равно $h(t) \cdot \tau + \xi$. Чтобы эта пачка не была отброшена в момент $t$, ее время ожидания не должно превышать значения $D$. Следовательно, $h(t) \leqslant d = \lfloor \frac{D-\xi}{\tau} \rfloor$. 

Благодаря тому, что числа $\tin$ и $\tres$ взаимно просты, цепь Маркова обладает свойством эргодичности. Таким образом, может быть найдено стационарное распределение вероятностей цепи Маркова. 

Выясним, в какие состояния и с какой вероятностью может перейти система из состояния $(h, m, n)$ за один шаг. Для этого отдельно рассмотрим несколько случаев.

\paragraph{1.} Пусть $h(t) < 0$, т.~е. в момент времени $t$ очередь пуста. 

\begin{itemize}
\item Если $h + \tres \ge 0$, то к моменту времени $t + 1$ в очередь поступит очередная пачка пакетов. Размер пачки является случайной величиной, подчиняющейся распределению условных вероятностей $p_{ji}$, где $j$ и $i$ --- размеры соседних пачек,  поэтому с вероятностью $p_{ni}$ система окажется в состоянии $(h + \tres, i, i)$. 

\item Если же $h + \tres < 0$, то к моменту времени $t + 1$ в очередь  не поступит ни одного пакета, т.~е. в этом случае система с вероятностью $1$ перейдет в состояние $(h + \tres, 0, n)$. 
\end{itemize}


\paragraph{2.} Пусть $h(t) \ge 0$ и $m(t) = 1$, т.~е. в головной пачке находится единственный пакет.

С вероятностью $1 - q$ происходит успешная передача пакета.

\begin{itemize}
\item Если при этом $h - \tin + \tres < 0$, то к моменту времени $t + 1$ на станцию не поступит ни одной пачки данных и система окажется в состоянии $(h - \tin + \tres, 0, n)$ с указанной выше вероятностью $1 - q$.
\item Если же $h - \tin + \tres \geqslant 0$, то в момент времени $t + 1$ очередь будет не пуста, и состояние системы будет определяться числом пакетов $i$ в головной пачке. Поскольку вероятность успешной попытки передачи пакета равна  $1 - q$, а пачка имеет размер $i$ с вероятностью $p_{ni}$, то система перейдет в состояние $(h - \tin + \tres, i, i)$, $i \in \{1, \ldots, M\}$, с вероятностью $(1 - q)p_{ni}$.
\end{itemize} 

С вероятностью $q$ попытка передачи пакета оказывается неудачной. 

\begin{itemize}
\item Если $h + \tres > d$, то к моменту времени $t + 1$ время ожидания данного пакета  превысит допустимое значение $D$, и этот пакет будет отброшен. Cистема c вероятностью $qp_{ni}$ перейдет в одно из состояний $(h - \tin + \tres, i, i)$, $i \in \{1, \ldots, M\}$.
\item Если $h + \tres \le d$, то к моменту времени $t + 1$ пакет не устареет, и будет предпринята следующая попытка передачи. Следовательно, система перейдет в состояние $(h + \tres, 1, n)$ с вероятностью $q$, а в состояние $(h + \tres, 0, n)$ с вероятностью $1 - q$.
\end{itemize} 

\paragraph{3.} Наконец, рассмотрим случай, когда $h(t) \geqslant 0$, $m(t) > 1$. 

С вероятностью $1 - q$ попытка передачи пакета была успешной. В этом случае в обслуживаемой пачке остается $m - 1$ пакет. 

\begin{itemize}
\item Если $h + \tres > d$, то к моменту времени $t + 1$ время ожидания этих пакетов превысит допустимое значение $D$, и все пакеты пачки будут отброшены. Система с вероятностью $(1-q)p_{ni}$ перейдет в одно из состояний $(h - \tin + \tres, i, i)$, $i \in \{1, \ldots, M\}$. 
\item Если $h + \tres \le d$, то система с вероятностью $1-q$ перейдет в состояние $(h + \tres, m - 1, n)$.
\end{itemize} 

Аналогично, в случае, когда попытка передачи не была успешной, система перейдет в состояние $(h - \tin + \tres, i, i)$ при  $h + \tres > d$ с вероятностью $q p_{ni}$ и в $(h + \tres, m, n)$ --- при $h + \tres \le d$ с вероятностью $q$.

\paragraph{Подводя итог,} получаем, что при выполнении соответствующих условий система  переходит из состояния $(h, m, n)$ в одно из следующих состояний:

\begin{tabular}{l l l l}
1)	&$(h + \tres, 0, n)$, 			&$\gamma = 1$, 	&при $m = 0$, $h < -\tres$; \\
2)  &$(h + \tres, i, i)$, 			&$\gamma = p_{ni}$, &при $m = 0$, $-\tres \leqslant h < 0$; \\
3)  &$(h + \tres - \tin, i, i)$, 	&$\gamma = (1 - q)p_{ni}$, &при $m = 1$, $\tin - \tres \leqslant h \leqslant d$; \\
4)  &$(h + \tres - \tin, 0, n)$, 	&$\gamma = 1 - q$, &при $m = 1$, $0 \leqslant h < \tin - \tres$; \\
5)  &$(h + \tres - \tin, i, i)$, 	&$\gamma = q p_{ni}$, &при $m = 1$, $d - \tres < h \leqslant d$; \\
6)  &$(h + \tres, m, n)$, 			&$\gamma = q$, &при $m > 0$, $0 \leqslant h \leqslant d - \tres$; \\
7)  &$(h + \tres - \tin, i, i)$,	&$\gamma = p_{ni}$, &при $m > 1$, $d - \tres < h \leqslant d$; \\
8)  &$(h + \tres, m - 1, n)$, &$\gamma = 1 - q$, &при $m > 1$, $0 \leqslant h \leqslant d - \tres$;
\end{tabular} \newline
где $i\in\{1,\ldots,M\}$ и $\gamma$ --- вероятность перехода. 

Уделим особое внимание переходу под номером~$7$. Он может осуществиться двумя способами. 

Первый способ заключается в том, что сначала с вероятностью $1 - q$ происходит успешная передача пакета, а потом с вероятностью $p_i$ переход в состояние $(h + \tres - \tin, i, i)$. Таким образом, вероятность перехода первым способом равна $(1 - q) p_{ni}$.

Второй способ заключается в том, что сначала с вероятностью $q$ возникает ошибка при передаче пакета, а затем с вероятностью $p_{ni}$ переход в состояние $(h + \tres - \tin, i, i)$. Получаем, что  вероятность перехода вторым способом равна $q p_{ni}$, а суммарная вероятность перехода $7$ равна $p_{ni}$. 

Однако между этими двумя способами реализации перехода есть существенное различие: при~первом способе отбрасывается $m - 1$ пакет, а при втором~--- $m$ пакетов. Данное обстоятельство окажется существенным при~подсчете значения $PLR$.  

\subsection{Определение PLR}
Упорядочив состояния в лексикографическом порядке и построив матрицу $P$ переходных вероятностей, найдем стационарные вероятности $\pi_{(h,m,n)}$ состояний $(h,m,n)$.

Зная станционарные вероятности $\pi_{(h,m, n)}$, найдем долю $PLR$ потерянных пакетов. Т.~к. пакеты отбрасываются только при превышении порога времени $D$ ожидания в очереди, потери пакетов могут происходить только при следующих переходах:
\begin{itemize}
\item[1)]переход из состояния $(h, m, n)$ в состояние \mbox{$(h - \tin + \tres, i, i)$} с вероятностью $q p_{ni}$ при условиях $m = 1$ и $d - \tres < h \leqslant d$ (отбрасывается $1$ пакет);
\item[2)]переход из состояния $(h, m, n)$ в состояние \mbox{$(h - \tin + \tres, i, i)$} с вероятностью $(1 - q)p_{ni}$ при условиях $m > 1$ и $d - \tres < h \leqslant d$ (отбрасывается $m - 1$ пакет); 
\item[3)]переход из состояния $(h, m, n)$ в состояние \mbox{$(h - \tin + \tres, i)$} с вероятностью $q p_{ni}$ при условиях $m > 1$ и $d - \tres < h \leqslant d$ (отбрасывается $m$ пакетов);
\end{itemize}

Таким образом, среднее число пакетов, отбрасываемых за один шаг модельного времени ($\Tres$), равно
$$
	\Idis = \sum\limits_{(h,m, n)\colon h > d - \tres} \left((m - 1)(1 - q)\pi_{(h,m,n)} + mq\pi_{(h,m,n)}\right) = 
	\sum\limits_{(h,m,n)\colon h > d - \tres} (m - 1 + q)\pi_{(h,m,n)}.
$$

Значение $PLR$ равно отношению среднего числа $\Idis$ пакетов, отброшенных за один шаг модельного времени, к среднему числу $\Iin$ пакетов, поступивших в очередь за то же время:
$$
PLR = \frac{\Idis}{\Iin} = \frac{1}{\frac{\Tres}{\Tin} \sum\limits_{i} i\cdot p_i} \cdot \left(\sum\limits_{(h,m,n)\colon h > d - \tres} (m - 1 + q)\pi_{(h,m,n)} \right).
$$

\section{Новая аналитическая модель} %????
\label{DynamicModel}
\subsection{Цепь Маркова}

Рассмотрим передачу пакетов следующим образом. Будем считать полностью известным входной поток $B$, т.~е. наперёд известен размер потока $l$ в пачках и размер каждой из них $B_i, i=\bar{1, l}$ в пакетах. Пусть состояние системы будет описываться тройкой параметров $(h,m,n)$, где $h$ и $m$ обозначают то же, что и в ранее описанной модели, с той лишь оговоркой, что $h$ выражается теперь в единицах времени (а не слотах, как в предыдущей модели), а $n$ равно номеру старшей пачки в потоке.

Назовём фронтом состояний системы в момент времени $t$ множество всех возможных состояний $(h(t), m(t), n(t))$, в которые система может попасть с ненулевой вероятностью. Будем рассматривать систему в моменты наступления периодов резервирования, в каждый из которых она имеет свой фронт $F_i$, где $i$ --- номер зарезервированного интервала в процессе передачи.
В общем случае $F(t)$ --- фронт состояний в момент времени $t$.
Будем считать, что если моменты резервирования и поступления пачки происходят в одно и то же время, то момент поступления пачки наступает раньше начала резервирования.
Начнём рассматривать систему в момент наступления первого резервирования. Таким образом, в начальный момент времени система находится в состоянии $(\xi, B_1, 1)$. Также каждому состоянию будет соответствовать вероятность попадания в него и среднее число отброшенных пакетов по его достижению.

Рассмотрим возможные переходы из состояния $(h,m,n)$ в фронте $F_i$ $i$-ого резервирования в состояния в фронте $F_{i + 1}$ $(i+1)$-ого резервирования, а также прирост среднего числа потерянных пакетов за эти переходы.

\paragraph{1.} $h < 0$, т.~е. очередь пуста.

\begin{itemize}
\item Если $h + \Tres \ge 0$, то к моменту времени $t + 1$ в очередь поступит очередная пачка пакетов. Размер пачки заранее определён, поэтому с вероятностью $1$ система окажется в состоянии $(h + \Tres, B_{i+1}, i+1)$.

\item Если же $h + \Tres < 0$, то к моменту времени $t + 1$ в очередь не поступит ни одного пакета, т.~е. в этом случае система с вероятностью $1$ перейдет в состояние $(h + \Tres, 0, i)$. 
\end{itemize}

В обоих случаях среднее число потерянных пакетов за переход равно нулю.
%%%W%%%%%%%%%%%%
\paragraph{2.} Пусть $h(t) \ge 0$ и $m(t) = 1$, т.~е. в головной пачке находится единственный пакет.

С вероятностью $1 - \qdet$ происходит успешная передача пакета.

\begin{itemize}
\item Если при этом $h - \Tin + \Tres < 0$, то к моменту времени $t + 1$ на станцию не поступит ни одной пачки данных и система окажется в состоянии $(h - \Tin + \Tres, 0, n)$.
\item Если же $h - \Tin + \Tres \geqslant 0$, то в момент времени $t + 1$ очередь будет не пуста, и состояние системы будет определяться числом пакетов в головной пачке с номером  $n + 1$, тогда система перейдет в состояние $(h - \Tin + \Tres, B_{n+1}, n+1)$.
\end{itemize}

В обоих этих случаях среднее число потерянных пакетов за переход равно нулю.

С вероятностью $\qdet$ происходит неуспешная передача пакета.

\begin{itemize}
\item Если $h + \Tres > d$, то к моменту времени $t + 1$ время ожидания данного пакета  превысит допустимое значение $D$, и этот пакет будет отброшен. Cистема c вероятностью $1$ перейдет в состояние $(h - \Tin + \Tres, B_{n+1}, n + 1)$. Среднее число потерянных пакетов в этом переходе равно $qdet$.
\item Если $h + \Tres \le d$, то к моменту времени $t + 1$ пакет не устареет, и будет предпринята следующая попытка передачи. Следовательно, система перейдет в состояние $(h + \Tres, 1, n)$ с вероятностью $1$. Среднее число потерянных пакетов в этом переходе равно нулю.
\end{itemize}

\paragraph{3.} Наконец, рассмотрим случай, когда $h(t) \geqslant 0$, $m(t) > 1$. 

\begin{itemize}
\item Если $h + \Tres > d$, то к моменту времени $t + 1$ время ожидания этих пакетов превысит допустимое значение $D$, и все пакеты пачки будут отброшены. Система с вероятностью $1$ перейдет в состояние $(h - \Tin + \Tres, B_{n+1}, n+1)$. Здесь среднее число потерянных пакетов за переход равно $m - 1 + q$.
\item Если $h + \Tres \le d$, то система с вероятностью $1 - q$ перейдет в состояние $(h + \Tres, m - 1, n)$, а с вероятностью $q$ перейдёт в состояние $(h + \Tres, m, n)$. 
\end{itemize}

Здесь в обоих переходах среднее число потерянных пакетов равно нулю.

После описания всевозможных переходов из фронта $F_i$ в фронт $F_{i+1}$ необходимо отнормировать все вероятности (в следующий фронт цепь попадает с вероятностью 1).

\paragraph{Подводя итог,} получаем, что при выполнении соответствующих условий система  переходит из состояния $(h, m, n)$ в одно из следующих состояний:

\begin{tabular}{l l l l l}
1)	&$(h + \Tres, 0, n)$, 			&$\gamma = 1$, &$\Delta N_{lost} = 0$,	&при $m = 0$, $h < -\Tres$; \\
2)  &$(h + \Tres, B_{n+1}, n+1)$, 			&$\gamma = 1$, &$\Delta N_{lost} = 0$, &при $m = 0$, $-\Tres \leqslant h < 0$; \\
3)  &$(h + \Tres - \Tin, B_{n+1}, n+1)$, 	&$\gamma = 1 - q$,  &$\Delta N_{lost} = 0$, &при $m = 1$, $\Tin - \Tres \leqslant h \leqslant d$; \\
4)  &$(h + \Tres - \Tin, 0, n)$, 	&$\gamma = 1 - q$,  &$\Delta N_{lost} = 0$, &при $m = 1$, $0 \leqslant h < \Tin - \Tres$; \\
5)  &$(h + \Tres - \Tin, B_{n+1}, n + 1)$, 	&$\gamma = q$,  &$\Delta N_{lost} = q$, &при $m = 1$, $d - \Tres < h \leqslant d$; \\
6)  &$(h + \Tres, m, n)$, 			&$\gamma = q$,  &$\Delta N_{lost} = 0$, &при $m > 0$, $0 \leqslant h \leqslant d - \Tres$; \\
7)  &$(h + \Tres - \Tin, B_{n+1}, n+1)$,	&$\gamma = 1$,  &$\Delta N_{lost} = m - 1 + q$, &при $m > 1$, $d - \Tres < h \leqslant d$; \\
8)  &$(h + \Tres, m - 1, n)$, &$\gamma = 1 - q$,  &$\Delta N_{lost} = 0$, &при $m > 1$, $0 \leqslant h \leqslant d - \Tres$;
\end{tabular} \newline
где $\gamma$ --- вероятность перехода,  $\Delta N_{lost}$ --- среднее число потерянных пакетов за переход.

\subsection{Определение PLR}
Зная всевозможные состояния в момент времени $t$ и соответствующее каждому из них среднее число отброшенных пакетов $N_{lost, (h, m, n)}$, можно вычислить среднее число отброшенных пакетов к моменту времени $t$:
$$
PLR(t) = \sum\limits_{(h,m,n) \in F(t)} \pi_{(h,m,n)} \cdot N_{lost, (h, m, n)}.
$$

С помощью вышеописанной модели можно найти параметры резервирования таким образом, чтобы удовлетворять QoS-требованиям, однако для этого необходима информация о всём видеопотоке, что на практике редко бывает. Поэтому следующая модель будет опираться лишь на то знание о видеопотоке, которое ей в действительности известна в соответствующий момент времени.

\subsection{Цепь Маркова для алгоритма}

Обобщим переходы новой модели на случай, когда видеопоток известен лишь частично.
Получив в момент времени $t$ на вход состояние $(h,m,n)$, можно найти ту часть видеопотока, которая системе на самом деле известна.
Действительно, это значит, что на данный момент в очереди находится пачка размера $m$ и возраста $h$, за ней в очереди пачка размера $n+1$ и возраста $h - \Tin$ и т. д. до пачки размера $n + k$ и возраста $h - k \cdot \Tin$, причём $h - k \cdot \Tin > 0 > h - (k + 1) \cdot \Tin$.
Система знает размер и время прихода уже пришедших пачек. Предполагая, что на размер следующей пачки будут влиять размеры предшествующих пачек, причём чем ближе по времени пачка, тем больше корреляция между размером такой пачки и размером следующей, будем считать, что распределение размеров пачек $\{ p_i \}_{i = 1}^{i = MAXSIZE}$ на момент времени $t$ (когда до этого момента всего пришло $s = \floor{\frac{t}{\Tin}}$ пачек, при этом среди них пачек размера $i$ было $s_i$ штук) равно

$$
p'_i = \sum\limits_{j = 1}^{s} \alpha\cdot   \hbox{The cool formula is here}
$$

Поэтому в случаях, когда видеопоток частично известен, математическая модель описывается, как в случае заранее заданного видеопотока до тех пор, а в дальнейшем они будут происходить с вероятностью, которая учитывает полученное распределение размеров пачек.

Рассмотрим возможные переходы из состояния $(h,m,n)$ в фронте $F_i$ $i$-ого резервирования в состояния в фронте $F_{i + 1}$ $(i+1)$-ого резервирования, а также прирост среднего числа потерянных пакетов за эти переходы. Опираясь на переходы в случае заранее заданного видеопотока, получим следующую таблицу переходов.

\begin{tabular}{l l l l}
1)	&$(h + \Tres, 0, n)$, 			&$\gamma = 1$, 	&при $m = 0$, $h < -\Tres$; \\
2)  &$(h + \Tres, i, n+1)$, 			&$\gamma = p_i$, &при $m = 0$, $-\Tres \leqslant h < 0$; \\
3)  &$(h + \Tres - \Tin, p_i, n+1)$, 	&$\gamma = (1 - q) p_i$, &при $m = 1$, $\Tin - \Tres \leqslant h \leqslant d$; \\
4)  &$(h + \Tres - \Tin, 0, n)$, 	&$\gamma = 1 - q$, &при $m = 1$, $0 \leqslant h < \Tin - \Tres$; \\
5)  &$(h + \Tres - \Tin, ш, n + 1)$, 	&$\gamma = q p_i$, &при $m = 1$, $d - \Tres < h \leqslant d$; \\
6)  &$(h + \Tres, m, n)$, 			&$\gamma = q$, &при $m > 0$, $0 \leqslant h \leqslant d - \Tres$; \\
7)  &$(h + \Tres - \Tin, i, n+1)$,	&$\gamma = p_i$, &при $m > 1$, $d - \Tres < h \leqslant d$; \\
8)  &$(h + \Tres, m - 1, n)$, &$\gamma = 1 - q$, &при $m > 1$, $0 \leqslant h \leqslant d - \Tres$;
\end{tabular} \newline
где $i\in\{1,\ldots, MAXSIZE\}$ и $\gamma$ --- вероятность перехода.

После описания всевозможных переходов из фронта $F_i$ в фронт $F_{i+1}$ необходимо отнормировать все вероятности (в следующий фронт цепь попадает с вероятностью 1).

\begin{algorithm}
\caption{Алгоритм динамического выбора периода резервирования}
\begin{algorithmic}[1]
\State{$i = 0$}
\State{$\Tres = default value$}
\State{$Distribution$}
\While{i < k}
	\State{Воспроизводим фрагмент № $i$ с заданным $\Tres$}
	\State{Получаем конечное состояние $(h,m,n)$}
	\State{Обновляем $Distribution$}
	\State{Загружаем $(h,m,n)$ и $Distribution$ в аналитическую модель, получив период резервирования для следующего окна $\Tres_{next}$}
	\State{$\Tres := \Tres_{next}$}
\EndWhile
\end{algorithmic}
\end{algorithm}


























%\input{BlockAck}
%\include{algorithm}
%\include{numerical}

\conclusion
%\input{conclusion}

%\appendix
%\input{app-a}

\bibliographystyle{ugost2008}
%\bibliography{thesis}

\end{document}