\subsection{Формальная постановка задачи}
\label{sec:problem_statement}


Рассмотрим передачу мультимедийного потока реального времени между двумя соседними станциями. Считается, что пакеты потока поступают в очередь пачками случайного размера (размер пачки принимает значение $i$ при учёте структуры уже известного на данный момент видеопотока $B(t)_det$ с вероятностью $\pi$, $i \in \{1, \ldots, MAXSIZE\}$) строго периодически с периодом $\Tin$, что соответствует передаче видеопотока с использованием протокола RTP. Далее такой поток будем называть периодическим, неординарным и коррелированным, то есть размер каждой следующей пачки зависит от его предыстории. QoS-требования представлены а) ограничением на время доставки пакета $\DQoS$ и б) ограничением на долю потерянных пакетов $\PLRQoS$. Если время нахождения пакета в очереди превышает $\DQoS$, то пакет отбрасывается, внося вклад в PLR.

Отправитель передает поток с использованием механизма комбинированного доступа, который работает следующим образом.
Отправитель устанавливает периодическое резервирование  с периодом $\Tres \leq \Tin$ и длительностью каждого зарезервированного интервала, рассчитанной ровно на одну попытку передачи пакета.
Отправитель всегда передает первый пакет из очереди.
%Для каждого пакета отправитель поддерживает счетчик передач: пакет может быть передан не более чем $\nretries$ раз.
%Спустя $\nretries$ попыток передачи пакет покидает очередь.
Также пакет покидает очередь, если был успешно передан, либо если его время нахождения в очереди превысило $\DQoS$.
Отправитель передает пакеты, используя либо детерминированный доступ (передавая пакеты в зарезервированных интервалах), либо случайный доступ. 
Передача с использованием случайного доступа ведется в промежутках между зарезервированными интервалами, т.е. в интервалах случайного доступа.
При этом пакет передается с использованием случайного доступа только тогда, когда он устаревает к началу следующего зарезерезвированного интервала.
В интервале случайного доступа можно совершить ограниченное количество попыток передач.
Время между двумя попытками передач в интервале случайного доступа считается  экспоненциально-распределенной случайной величиной с параметром $\lambda$. 

Вероятность неуспешной передачи пакета внутри зарезервированного интервала равна $\qdet$, а в течение интервала случайного доступа --- $\qran \geq \qdet$. 

При таком использовании механизма комбинированного доступа основной вопрос состоит в том, как выбрать параметр $\Tres$, чтобы выполнить QoS-требования, используя при этом как можно меньше канального времени. Для ответа на этот вопрос была разработана математическая модель рассматриваемого процесса передачи. Эта модель позволяет найти PLR как функцию от $\Tres$ и по зависимости $PLR(\Tres)$ найти искомый параметр.

Также был разработан алгоритм нахождения максимального значения $\Tres$, при котором выполняются QoS-требования, в динамическом режиме, учитывающий состояние очереди в моменты принятия решения, а также распределение размеров пришедших к этому моменту пачек.

%В течении зарезервированного интервала передача пакета происходит неудачно с вероятностью $\perm$. Если для передачи используется метод случайного доступа (пакет передается за пределами зарезервированного интервала), то передача происходит неудачно с вероятностью $\pere$. 

%Мы также предполагаем, что в какой бы момент до следующего зарезервированного интервала пакет ни устарел, времени на совершение всех необходимых передач с использованием метода случайного доступа будет достаточно. 

%Наконец, \textit{пакет отбрасывается, если он не был передан успешно, только если его счетчик количества попыток передач достигает значения $\nretries$}.         



