\subsection{Обзор существующих работ}
\label{sec:related_papers}

Часть работ, рассматривающих совместное использование случайного и детерминированного доступа, посвящены тому, как именно распределять ресурсы между этими механизмами. Например,~\cite{siris2006resource} показывает, как разделить ресурсы между EDCA и HCCA для достижения максимальной пропускной способности в сетях IEEE 802.11e. Сосуществование двух механизмов доступа также рассмотрено в~\cite{Krasilov2013} в рамках использования EDCA и MCCA в сетях Wi-Fi Mesh. \cite{Krasilov2013}~показывает, как количество зарезервированного времени влияет на пропускную способность EDCA. Однако в данных работах не рассмотрены ни совместное использование двух механизмов доступа для передачи одного потока данных, ни вопрос выполнения QoS-требований. 

Что касается комбинированного использования двух механизмов доступа, то на эту тему существует всего лишь несколько работ, и все они в основном рассматривают HEMM. К примеру, \cite{kuan2007utilization} рассматривает ситуацию, в которой несколько станций передают в режиме насыщения, используя как EDCA, так и HCCA. Показано, что чем больше данных передается с помощью HCCA, тем больше доля времени, потраченного на успешные передачи. \cite{Ng2012}~предлагает использовать марковский процесс принятия решений для координирования HCCA и EDCA в сетях  IEEE 802.11e. Однако \cite{Ng2012}~не учитывает QoS-требования и ставит задачей лишь увеличение доли времени, потраченного на успешные передачи. Кроме того, ни  \cite{kuan2007utilization} ни \cite{Ng2012} не изучают применение HEMM для передачи потока с QoS-требованиями.

Влияние HEMM на выполнение QoS-требований рассмотрено в~\cite{ruscelli2012enhancement}, где представлены описание и анализ EDCA, HCCA и HEMM. \cite{ruscelli2012enhancement} предлагает улучшенный алгоритм управления доступом, который применяется для HEMM и может быть использован для улучшения показателей практически любого HCCA планировщика, что подтверждено имитационным моделированием. При этом не было разработано математической модели, которая могла бы быть использована для настройки параметров комбинированного доступа для поддержки QoS-требований.

В работах (таких-то) уже представлены математические модели, по которым можно найти параметры резервирования, при которых выполнены QoS-требования, а потребление канальных ресурсов минимально. Однако эти модели плохо сходятся в областях, где PLR достигает значений $10^{-4} \div 10^{-3}$, считающиеся на сегодняшний день наиболее актуальными.