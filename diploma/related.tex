\subsection{Обзор существующих работ}
\label{sec:related_papers}

Часть работ, рассматривающих совместное использование случайного и детерминированного доступа, посвящены тому, как именно распределять ресурсы между этими механизмами. Например,~\cite{siris2006resource} показывает, как разделить ресурсы между EDCA и HCCA для достижения максимальной пропускной способности в сетях IEEE 802.11e. Сосуществование двух механизмов доступа также рассмотрено в~\cite{Krasilov2013} в рамках использования EDCA и MCCA в сетях Wi-Fi Mesh. \cite{Krasilov2013}~показывает, как количество зарезервированного времени влияет на пропускную способность EDCA. Однако в данных работах не рассмотрены ни совместное использование двух механизмов доступа для передачи одного потока данных, ни вопрос выполнения QoS-требований. 

%A number of papers devoted to both random and deterministic access study how to share resources between these methods.
%For example,~\cite{siris2006resource} gives recommendation on how to share resources between EDCA and HCCA to achieve the maximum overall throughput in IEEE 802.11e networks. The coexistence of the methods is also studied in~\cite{Krasilov2013} in terms of EDCA and MCCA in Wi-Fi Mesh networks. \cite{Krasilov2013}~shows how the amount of the reserved channel time influences the throughput of EDCA. But no QoS provisioning or joint usage is considered in both~\cite{siris2006resource} and~\cite{Krasilov2013}.

Что касается комбинированного использования двух механизмов доступа, то на эту тему существует всего лишь несколько работ, и все они в основном рассматривают HEMM. К примеру, \cite{kuan2007utilization} рассматривает ситуацию, в которой несколько станций передают в режиме насыщения, используя как EDCA, так и HCCA. Показано, что чем больше данных передается с помощью HCCA, тем больше доля времени, потраченного на успешные передачи. \cite{Ng2012}~предлагает использовать марковский процесс принятия решений для координирования HCCA и EDCA в сетях  IEEE 802.11e. Однако \cite{Ng2012}~не учитывает QoS-требования и ставит задачей лишь увеличение доли времени, потраченного на успешные передачи. Кроме того, ни  \cite{kuan2007utilization} ни \cite{Ng2012} не изучают применение HEMM для передачи потока с QoS-требованиями.

%As for the joint access, only a few studies exist so far and they mainly consider HCCA-EDCA Mixed Mode (HEMM). For example, \cite{kuan2007utilization} considers a scenario where a set of STAs transmit saturated traffic using both EDCA and HCCA. It is shown that the more data is transmitted with HCCA the higher is the channel utilization, that is, the percentage of time spend for successful transmissions. \cite{Ng2012}~proposes to use a Markov decision process for coordinating HCCA and EDCA in IEEE 802.11e networks. However, \cite{Ng2012}~considers no QoS requirements and aims only to improve the channel utilization. 
%In this paper, we propose a joint access method. But neither \cite{kuan2007utilization} nor \cite{Ng2012} study the ability of HEMM to transmit QoS-sensitive data. 

Влияние HEMM на выполнение QoS-требований рассмотрено в~\cite{ruscelli2012enhancement}, где представлены описание и анализ EDCA, HCCA и HEMM. \cite{ruscelli2012enhancement} предлагает улучшенный алгоритм управления доступом, который применяется для HEMM и может быть использован для улучшения показателей практически любого HCCA планировщика, что подтверждено имитационным моделированием. При этом не было разработано математической модели, которая могла бы быть использована для настройки параметров комбинированного доступа для поддержки QoS-требований.  

%The influence of HEMM on QoS provisioning is studied in~\cite{ruscelli2012enhancement} where a good description/analysis of EDCA, HCCA as well as HEMM can be found. \cite{ruscelli2012enhancement} proposes an enhanced admission control algorithm which accounts for HEMM and can be used to improve performance of almost any HCCA scheduler as shown by simulation. However, no analytical model, which can be used to study how joint access is efficient for supporting QoS, has been developed so far. Realizing this gap, we try to fill it in the current paper.
