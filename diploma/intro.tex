%\textbf{Актуальность задачи.} С ростом популярности сервисов реального времени растут как объемы генерируемых ими мультимедийных данных, так и требования пользователей к качеству обслуживания таких данных, QoS-требования (Quality of Service). QoS-требования данных реального времени включают в себя не только минимальную пропускную способность, но и ограничения на задержку передачи и долю потерянных пакетов PLR (Packet Loss Ratio). Выполнение QoS-требований при передаче мультимедийных данных с минимальными затратами канальных ресурсов является одной из наиболее актуальных задач в области беспроводных коммуникаций.

\textbf{Актуальность задачи.} В течение десятилетий растёт популярность сервисов реального времени, а вместе с ней неизбежно растут размеры генерируемых ими мультимедийных данных, а также ужесточаются требования пользователей к качеству обслуживания таких данных, QoS-требования (Quality of Service). QoS-требования данных реального времени включают в себя минимальную пропускную способность. Более того, к QoS-требованиям относятся ограничения на задержку передачи и долю потерянных пакетов PLR (Packet Loss Ratio). Передача мультимедийных данных с минимальными затратами канальных ресурсов и выполнением QoS-требований является одной из наиболее актуальных задач в области беспроводных коммуникаций.

В сетях Wi-Fi возможность выполнения QoS-требований зависит от используемых механизмов доступа к каналу. При использовании механизмов \textit{детерминированного} доступа станции передают в заранее зарезервированных ими интервалах времени. С одной стороны, передачи в зарезервированных интервалах защищены от коллизий, поскольку станции не ведут передачу в чужих интервалах. С другой стороны, изменение зарезервированных интервалов занимает существенное время, затрачиваемое на согласование новых параметров зарезервированных интервалов с соседними станциями. Такие задержки могут быть особенно критичны в случае передачи мультимедийных потоков реального времени, кратковременные всплески интенсивности которых вынуждают либо изменять параметры зарезервированных интервалов, либо резервировать канальные ресурсы с избытком.

При использовании механизмов \textit{случайного} доступа станции выбирают случайные моменты времени для передачи данных. Это обеспечивает б\'{о}льшую гибкость по сравнению с детерминированным доступом, поскольку при их использовании для передачи данных можно попытаться получить доступ к каналу после того, как он оказался свободным в течение некоторого промежутка времени, а не ждать ближайшего зарезервированного интервала. По этой причине механизмы случайного доступа эффективны для поглощения всплесков интенсивности передаваемых потоков. В то же время таким механизмам свойственны частые коллизии (особенно в плотных сетях) и случайные задержки при попытках получить доступ к каналу. В результате механизмы случайного доступа не могут гарантировать выполнение QoS-требований при передаче данных реального времени. 

Механизм детерминированного доступа может быть улучшен за счет гибкости случайного доступа,  если использовать эти два вида доступа совместно в составе некоторого механизма \textit{комбинированного} доступа: основная часть потока может быть передана внутри зарезервированных интервалов, в то время как для поглощения всплесков интенсивности передаваемого потока дополнительно задействуется случайный доступ. Комбинированный доступ обещает значительные выгоды как с точки зрения возможности выполнения QoS-требований (за счет передачи внутри защищенных зарезервированных интервалов), так и с точки зрения объема используемого канального времени (не нужно резервировать время с расчетом на наибольшую возможную интенсивность входного потока, а только для передачи его постоянной составляющей). Основной вопрос заключается в том, как именно нужно использовать механизмы комбинированного доступа при передаче данных реального времени.
 
\textbf{Задачей данной работы} является анализ использования механизма комбинированного доступа для передачи мультимедийных данных реального времени и разработка метода выбора таких его параметров, при которых выполнены QoS-требования к передаче потока, а потребление канальных ресурсов минимально. Для решения данной задачи в работе разработана математическая модель передачи мультимедийного потока с использованием  механизма комбинированного доступа. Данная модель позволяет сравнить механизм комбинированного доступа с механизмом детерминированного доступа в плане потребления канальных ресурсов с необходимыми для выполнения QoS-требованиями к передаче потока. При построении математической модели были использованы такие \textbf{методы исследования}, как теория вероятностей и комбинаторный анализ.    

\textbf{Научная новизна} данной работы заключается в том, что в ней разработана новая математическая модель процесса передачи реального мультимедийного потока с использованием механизма комбинированного доступа с учетом QoS-требований и возможных ошибок при передаче пакетов.

\textbf{Положение, выносимое на защиту.} В данной работе была разработана новая математическая модель процесса передачи реального мультимедийного потока с использованием механизма комбинированного доступа. Разработанная модель позволяет найти такие параметры комбинированного доступа, при которых выполнены QoS-требования передаваемого потока, а потребление канальных ресурсов минимально.

\textbf{Данная работа построена следующим образом.} В разделе~\ref{sec:problem_title} приводится краткое описание рассматриваемых в работе механизмов доступа, обзор существующих работ и  формальная постановка задачи. Раздел~\ref{sec:math_model} посвящен построению математической модели рассматриваемого процесса передачи. В разделе~\ref{sec:numerical} разработанная модель применяется для выбора параметров передачи, демонстрируется выгода от использования предложенного механизма доступа по сравнению с использованием механизма детерминированного доступа. Наконец, итоги работы приводятся в заключении.