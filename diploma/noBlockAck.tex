\chapter{Математическая модель одиночной передачи пакетов}
\label{chapter:model_noBlockAck}

При анализе новой математической модели было произведено её сравнение с уже существующими моделями. Проведём их детальное описание.

\section{Аналитическая модель передачи}
\label{analyticalModel1}
\subsection{Цепь Маркова}

Представим $\Tin/\Tres$ в виде несократимой дроби $\tin/\tres$, где $\tin, \tres\in\mathbb{N}$. Назовем слотом интервал времени длины $$\tau = \frac{\Tin}{\tin} \equiv \frac{\Tres}{\tres}.$$ Разобьем непрерывную временную шкалу на слоты таким образом, чтобы начало каждого MCCAOP совпадало с началом некоторого слота -- см.~рис~\ref{fig:time}.

Процесс передачи пакетов с помощью механизма MCCA может быть описан цепью Маркова с дискретным временем, единица которого равна периоду резервирований.

%\begin{figure}[t]
%\centering{\includegraphics[width=0.7\linewidth]{pics/time1}}
%	\caption{\label{fig:time}Слотированное время}
%\end{figure}


В каждый момент времени $t$ состояние системы будем описывать парой целых чисел $(h(t), m(t), n(t))$. Если $h(t) \geqslant 0$, то очередь не пуста, и $h(t)$ равно числу полных слотов, которые головная (самая старшая) пачка пакетов провела в очереди, $m(t)$ равно числу пакетов в этой пачке, а $n(t)$ равно начальному числу пакетов в головной пачке. Если $h(t) < 0$, то очередь пуста, и $|h(t)|$ равно времени до прибытия новой пачки пакетов, выраженному в слотах и округленному вниз; при этом $m=0$. Таким образом, состояние системы в каждый моменты времени $t$ характеризуется текущим числом пакетов в головной пачке, начальным числом пакетов  в головной пачке и временем, в течение которого пакеты этой пачки ожидают передачи. Используемые обозначения состояния системы позволяют определить количество пачек пакетов в очереди:  $\lfloor h(t)/\tin \rfloor + 1$, но не позволяют определить длины всех пачек, кроме головной. После того, как все пакеты головной пачки были переданы или отброшены, определяется число пакетов в пачке, ставшей головной.
  
Минимальное значение $h(t)$ равно $\tres-\tin$. Оно достигается в момент времени $t$, если пачка из одного пакета пребывает в пустую очередь непосредственно перед моментом $t - 1$, и единственный пакет пачки успешно передается с первой попытки.   

Теперь найдем максимальное возможное значение $h(t)$. Для этого обозначим через $\xi$ время между поступлением пачки пакетов в очередь и началом следующего слота, $0 \leqslant \xi < \tau$. В силу того, что время $\Tin$ равно целому числу $\tin$ слотов, значение величины $\xi$ одинаково для всех пачек. Таким образом, к моменту времени $t$ время ожидания в очереди пакетов головной пачки равно $h(t) \cdot \tau + \xi$. Чтобы эта пачка не была отброшена в момент $t$, ее время ожидания не должно превышать значения $D$. Следовательно, $h(t) \leqslant d = \lfloor \frac{D-\xi}{\tau} \rfloor$. 

Благодаря тому, что числа $\tin$ и $\tres$ взаимно просты, цепь Маркова обладает свойством эргодичности. Таким образом, может быть найдено стационарное распределение вероятностей цепи Маркова. 

Выясним, в какие состояния и с какой вероятностью может перейти система из состояния $(h, m, n)$ за один шаг. Для этого отдельно рассмотрим несколько случаев.

\paragraph{1.} Пусть $h(t) < 0$, т.~е. в момент времени $t$ очередь пуста. 

\begin{itemize}
\item Если $h + \tres \ge 0$, то к моменту времени $t + 1$ в очередь поступит очередная пачка пакетов. Размер пачки является случайной величиной, подчиняющейся распределению условных вероятностей $p_{ji}$, где $j$ и $i$ --- размеры соседних пачек,  поэтому с вероятностью $p_{ni}$ система окажется в состоянии $(h + \tres, i, i)$. 

\item Если же $h + \tres < 0$, то к моменту времени $t + 1$ в очередь  не поступит ни одного пакета, т.~е. в этом случае система с вероятностью $1$ перейдет в состояние $(h + \tres, 0, n)$. 
\end{itemize}


\paragraph{2.} Пусть $h(t) \ge 0$ и $m(t) = 1$, т.~е. в головной пачке находится единственный пакет.

С вероятностью $1 - q$ происходит успешная передача пакета.

\begin{itemize}
\item Если при этом $h - \tin + \tres < 0$, то к моменту времени $t + 1$ на станцию не поступит ни одной пачки данных и система окажется в состоянии $(h - \tin + \tres, 0, n)$ с указанной выше вероятностью $1 - q$.
\item Если же $h - \tin + \tres \geqslant 0$, то в момент времени $t + 1$ очередь будет не пуста, и состояние системы будет определяться числом пакетов $i$ в головной пачке. Поскольку вероятность успешной попытки передачи пакета равна  $1 - q$, а пачка имеет размер $i$ с вероятностью $p_{ni}$, то система перейдет в состояние $(h - \tin + \tres, i, i)$, $i \in \{1, \ldots, M\}$, с вероятностью $(1 - q)p_{ni}$.
\end{itemize} 

С вероятностью $q$ попытка передачи пакета оказывается неудачной. 

\begin{itemize}
\item Если $h + \tres > d$, то к моменту времени $t + 1$ время ожидания данного пакета  превысит допустимое значение $D$, и этот пакет будет отброшен. Cистема c вероятностью $qp_{ni}$ перейдет в одно из состояний $(h - \tin + \tres, i, i)$, $i \in \{1, \ldots, M\}$.
\item Если $h + \tres \le d$, то к моменту времени $t + 1$ пакет не устареет, и будет предпринята следующая попытка передачи. Следовательно, система перейдет в состояние $(h + \tres, 1, n)$ с вероятностью $q$, а в состояние $(h + \tres, 0, n)$ с вероятностью $1 - q$.
\end{itemize} 

\paragraph{3.} Наконец, рассмотрим случай, когда $h(t) \geqslant 0$, $m(t) > 1$. 

С вероятностью $1 - q$ попытка передачи пакета была успешной. В этом случае в обслуживаемой пачке остается $m - 1$ пакет. 

\begin{itemize}
\item Если $h + \tres > d$, то к моменту времени $t + 1$ время ожидания этих пакетов превысит допустимое значение $D$, и все пакеты пачки будут отброшены. Система с вероятностью $(1-q)p_{ni}$ перейдет в одно из состояний $(h - \tin + \tres, i, i)$, $i \in \{1, \ldots, M\}$. 
\item Если $h + \tres \le d$, то система с вероятностью $1-q$ перейдет в состояние $(h + \tres, m - 1, n)$.
\end{itemize} 

Аналогично, в случае, когда попытка передачи не была успешной, система перейдет в состояние $(h - \tin + \tres, i, i)$ при  $h + \tres > d$ с вероятностью $q p_{ni}$ и в $(h + \tres, m, n)$ --- при $h + \tres \le d$ с вероятностью $q$.

\paragraph{Подводя итог,} получаем, что при выполнении соответствующих условий система  переходит из состояния $(h, m, n)$ в одно из следующих состояний:

\begin{tabular}{l l l l}
1)	&$(h + \tres, 0, n)$, 			&$\gamma = 1$, 	&при $m = 0$, $h < -\tres$; \\
2)  &$(h + \tres, i, i)$, 			&$\gamma = p_{ni}$, &при $m = 0$, $-\tres \leqslant h < 0$; \\
3)  &$(h + \tres - \tin, i, i)$, 	&$\gamma = (1 - q)p_{ni}$, &при $m = 1$, $\tin - \tres \leqslant h \leqslant d$; \\
4)  &$(h + \tres - \tin, 0, n)$, 	&$\gamma = 1 - q$, &при $m = 1$, $0 \leqslant h < \tin - \tres$; \\
5)  &$(h + \tres - \tin, i, i)$, 	&$\gamma = q p_{ni}$, &при $m = 1$, $d - \tres < h \leqslant d$; \\
6)  &$(h + \tres, m, n)$, 			&$\gamma = q$, &при $m > 0$, $0 \leqslant h \leqslant d - \tres$; \\
7)  &$(h + \tres - \tin, i, i)$,	&$\gamma = p_{ni}$, &при $m > 1$, $d - \tres < h \leqslant d$; \\
8)  &$(h + \tres, m - 1, n)$, &$\gamma = 1 - q$, &при $m > 1$, $0 \leqslant h \leqslant d - \tres$;
\end{tabular} \newline
где $i\in\{1,\ldots,M\}$ и $\gamma$ --- вероятность перехода. 

Уделим особое внимание переходу под номером~$7$. Он может осуществиться двумя способами. 

Первый способ заключается в том, что сначала с вероятностью $1 - q$ происходит успешная передача пакета, а потом с вероятностью $p_i$ переход в состояние $(h + \tres - \tin, i, i)$. Таким образом, вероятность перехода первым способом равна $(1 - q) p_{ni}$.

Второй способ заключается в том, что сначала с вероятностью $q$ возникает ошибка при передаче пакета, а затем с вероятностью $p_{ni}$ переход в состояние $(h + \tres - \tin, i, i)$. Получаем, что  вероятность перехода вторым способом равна $q p_{ni}$, а суммарная вероятность перехода $7$ равна $p_{ni}$. 

Однако между этими двумя способами реализации перехода есть существенное различие: при~первом способе отбрасывается $m - 1$ пакет, а при втором~--- $m$ пакетов. Данное обстоятельство окажется существенным при~подсчете значения $PLR$.  

\subsection{Определение PLR}
Упорядочив состояния в лексикографическом порядке и построив матрицу $P$ переходных вероятностей, найдем стационарные вероятности $\pi_{(h,m,n)}$ состояний $(h,m,n)$.

Зная станционарные вероятности $\pi_{(h,m, n)}$, найдем долю $PLR$ потерянных пакетов. Т.~к. пакеты отбрасываются только при превышении порога времени $D$ ожидания в очереди, потери пакетов могут происходить только при следующих переходах:
\begin{itemize}
\item[1)]переход из состояния $(h, m, n)$ в состояние \mbox{$(h - \tin + \tres, i, i)$} с вероятностью $q p_{ni}$ при условиях $m = 1$ и $d - \tres < h \leqslant d$ (отбрасывается $1$ пакет);
\item[2)]переход из состояния $(h, m, n)$ в состояние \mbox{$(h - \tin + \tres, i, i)$} с вероятностью $(1 - q)p_{ni}$ при условиях $m > 1$ и $d - \tres < h \leqslant d$ (отбрасывается $m - 1$ пакет); 
\item[3)]переход из состояния $(h, m, n)$ в состояние \mbox{$(h - \tin + \tres, i)$} с вероятностью $q p_{ni}$ при условиях $m > 1$ и $d - \tres < h \leqslant d$ (отбрасывается $m$ пакетов);
\end{itemize}

Таким образом, среднее число пакетов, отбрасываемых за один шаг модельного времени ($\Tres$), равно
$$
	\Idis = \sum\limits_{(h,m, n)\colon h > d - \tres} \left((m - 1)(1 - q)\pi_{(h,m,n)} + mq\pi_{(h,m,n)}\right) = 
	\sum\limits_{(h,m,n)\colon h > d - \tres} (m - 1 + q)\pi_{(h,m,n)}.
$$

Значение $PLR$ равно отношению среднего числа $\Idis$ пакетов, отброшенных за один шаг модельного времени, к среднему числу $\Iin$ пакетов, поступивших в очередь за то же время:
$$
PLR = \frac{\Idis}{\Iin} = \frac{1}{\frac{\Tres}{\Tin} \sum\limits_{i} i\cdot p_i} \cdot \left(\sum\limits_{(h,m,n)\colon h > d - \tres} (m - 1 + q)\pi_{(h,m,n)} \right).
$$

\section{Новая аналитическая модель} %????
\label{DynamicModel}
\subsection{Цепь Маркова}

Рассмотрим передачу пакетов следующим образом. Будем считать полностью известным входной поток $B$, т.~е. наперёд известен размер потока $l$ в пачках и размер каждой из них $B_i, i=\bar{1, l}$ в пакетах. Пусть состояние системы будет описываться тройкой параметров $(h,m,n)$, где $h$ и $m$ обозначают то же, что и в ранее описанной модели, с той лишь оговоркой, что $h$ выражается теперь в единицах времени (а не слотах, как в предыдущей модели), а $n$ равно номеру старшей пачки в потоке.

Назовём фронтом состояний системы в момент времени $t$ множество всех возможных состояний $(h(t), m(t), n(t))$, в которые система может попасть с ненулевой вероятностью. Будем рассматривать систему в моменты наступления периодов резервирования, в каждый из которых она имеет свой фронт $F_i$, где $i$ --- номер зарезервированного интервала в процессе передачи.
В общем случае $F(t)$ --- фронт состояний в момент времени $t$.
Будем считать, что если моменты резервирования и поступления пачки происходят в одно и то же время, то момент поступления пачки наступает раньше начала резервирования.
Начнём рассматривать систему в момент наступления первого резервирования. Таким образом, в начальный момент времени система находится в состоянии $(\xi, B_1, 1)$. Также каждому состоянию будет соответствовать вероятность попадания в него и среднее число отброшенных пакетов по его достижению.

Рассмотрим возможные переходы из состояния $(h,m,n)$ в фронте $F_i$ $i$-ого резервирования в состояния в фронте $F_{i + 1}$ $(i+1)$-ого резервирования, а также прирост среднего числа потерянных пакетов за эти переходы.

\paragraph{1.} $h < 0$, т.~е. очередь пуста.

\begin{itemize}
\item Если $h + \Tres \ge 0$, то к моменту времени $t + 1$ в очередь поступит очередная пачка пакетов. Размер пачки заранее определён, поэтому с вероятностью $1$ система окажется в состоянии $(h + \Tres, B_{i+1}, i+1)$.

\item Если же $h + \Tres < 0$, то к моменту времени $t + 1$ в очередь не поступит ни одного пакета, т.~е. в этом случае система с вероятностью $1$ перейдет в состояние $(h + \Tres, 0, i)$. 
\end{itemize}

В обоих случаях среднее число потерянных пакетов за переход равно нулю.
%%%W%%%%%%%%%%%%
\paragraph{2.} Пусть $h(t) \ge 0$ и $m(t) = 1$, т.~е. в головной пачке находится единственный пакет.

С вероятностью $1 - \qdet$ происходит успешная передача пакета.

\begin{itemize}
\item Если при этом $h - \Tin + \Tres < 0$, то к моменту времени $t + 1$ на станцию не поступит ни одной пачки данных и система окажется в состоянии $(h - \Tin + \Tres, 0, n)$.
\item Если же $h - \Tin + \Tres \geqslant 0$, то в момент времени $t + 1$ очередь будет не пуста, и состояние системы будет определяться числом пакетов в головной пачке с номером  $n + 1$, тогда система перейдет в состояние $(h - \Tin + \Tres, B_{n+1}, n+1)$.
\end{itemize}

В обоих этих случаях среднее число потерянных пакетов за переход равно нулю.

С вероятностью $\qdet$ происходит неуспешная передача пакета.

\begin{itemize}
\item Если $h + \Tres > d$, то к моменту времени $t + 1$ время ожидания данного пакета  превысит допустимое значение $D$, и этот пакет будет отброшен. Cистема c вероятностью $1$ перейдет в состояние $(h - \Tin + \Tres, B_{n+1}, n + 1)$. Среднее число потерянных пакетов в этом переходе равно $qdet$.
\item Если $h + \Tres \le d$, то к моменту времени $t + 1$ пакет не устареет, и будет предпринята следующая попытка передачи. Следовательно, система перейдет в состояние $(h + \Tres, 1, n)$ с вероятностью $1$. Среднее число потерянных пакетов в этом переходе равно нулю.
\end{itemize}

\paragraph{3.} Наконец, рассмотрим случай, когда $h(t) \geqslant 0$, $m(t) > 1$. 

\begin{itemize}
\item Если $h + \Tres > d$, то к моменту времени $t + 1$ время ожидания этих пакетов превысит допустимое значение $D$, и все пакеты пачки будут отброшены. Система с вероятностью $1$ перейдет в состояние $(h - \Tin + \Tres, B_{n+1}, n+1)$. Здесь среднее число потерянных пакетов за переход равно $m - 1 + q$.
\item Если $h + \Tres \le d$, то система с вероятностью $1 - q$ перейдет в состояние $(h + \Tres, m - 1, n)$, а с вероятностью $q$ перейдёт в состояние $(h + \Tres, m, n)$. 
\end{itemize}

Здесь в обоих переходах среднее число потерянных пакетов равно нулю.

После описания всевозможных переходов из фронта $F_i$ в фронт $F_{i+1}$ необходимо отнормировать все вероятности (в следующий фронт цепь попадает с вероятностью 1).

\paragraph{Подводя итог,} получаем, что при выполнении соответствующих условий система  переходит из состояния $(h, m, n)$ в одно из следующих состояний:

\begin{tabular}{l l l l l}
1)	&$(h + \Tres, 0, n)$, 			&$\gamma = 1$, &$\Delta N_{lost} = 0$,	&при $m = 0$, $h < -\Tres$; \\
2)  &$(h + \Tres, B_{n+1}, n+1)$, 			&$\gamma = 1$, &$\Delta N_{lost} = 0$, &при $m = 0$, $-\Tres \leqslant h < 0$; \\
3)  &$(h + \Tres - \Tin, B_{n+1}, n+1)$, 	&$\gamma = 1 - q$,  &$\Delta N_{lost} = 0$, &при $m = 1$, $\Tin - \Tres \leqslant h \leqslant d$; \\
4)  &$(h + \Tres - \Tin, 0, n)$, 	&$\gamma = 1 - q$,  &$\Delta N_{lost} = 0$, &при $m = 1$, $0 \leqslant h < \Tin - \Tres$; \\
5)  &$(h + \Tres - \Tin, B_{n+1}, n + 1)$, 	&$\gamma = q$,  &$\Delta N_{lost} = q$, &при $m = 1$, $d - \Tres < h \leqslant d$; \\
6)  &$(h + \Tres, m, n)$, 			&$\gamma = q$,  &$\Delta N_{lost} = 0$, &при $m > 0$, $0 \leqslant h \leqslant d - \Tres$; \\
7)  &$(h + \Tres - \Tin, B_{n+1}, n+1)$,	&$\gamma = 1$,  &$\Delta N_{lost} = m - 1 + q$, &при $m > 1$, $d - \Tres < h \leqslant d$; \\
8)  &$(h + \Tres, m - 1, n)$, &$\gamma = 1 - q$,  &$\Delta N_{lost} = 0$, &при $m > 1$, $0 \leqslant h \leqslant d - \Tres$;
\end{tabular} \newline
где $\gamma$ --- вероятность перехода,  $\Delta N_{lost}$ --- среднее число потерянных пакетов за переход.

\subsection{Определение PLR}
Зная всевозможные состояния в момент времени $t$ и соответствующее каждому из них среднее число отброшенных пакетов $N_{lost, (h, m, n)}$, можно вычислить среднее число отброшенных пакетов к моменту времени $t$:
$$
PLR(t) = \sum\limits_{(h,m,n) \in F(t)} \pi_{(h,m,n)} \cdot N_{lost, (h, m, n)}.
$$

С помощью вышеописанной модели можно найти параметры резервирования таким образом, чтобы удовлетворять QoS-требованиям, однако для этого необходима информация о всём видеопотоке, что на практике редко бывает. Поэтому следующая модель будет опираться лишь на то знание о видеопотоке, которое ей в действительности известна в соответствующий момент времени.

\subsection{Цепь Маркова для алгоритма}

Обобщим переходы новой модели на случай, когда видеопоток известен лишь частично.
Получив в момент времени $t$ на вход состояние $(h,m,n)$, можно найти ту часть видеопотока, которая системе на самом деле известна.
Действительно, это значит, что на данный момент в очереди находится пачка размера $m$ и возраста $h$, за ней в очереди пачка размера $n+1$ и возраста $h - \Tin$ и т. д. до пачки размера $n + k$ и возраста $h - k \cdot \Tin$, причём $h - k \cdot \Tin > 0 > h - (k + 1) \cdot \Tin$.
Система знает размер и время прихода уже пришедших пачек. Предполагая, что на размер следующей пачки будут влиять размеры предшествующих пачек, причём чем ближе по времени пачка, тем больше корреляция между размером такой пачки и размером следующей, будем считать, что распределение размеров пачек $\{ p_i \}_{i = 1}^{i = MAXSIZE}$ на момент времени $t$ (когда до этого момента всего пришло $s = \floor{\frac{t}{\Tin}}$ пачек, при этом среди них пачек размера $i$ было $s_i$ штук) равно

$$
p'_i = \sum\limits_{j = 1}^{s} \alpha\cdot   \hbox{The cool formula is here}
$$

Поэтому в случаях, когда видеопоток частично известен, математическая модель описывается, как в случае заранее заданного видеопотока до тех пор, а в дальнейшем они будут происходить с вероятностью, которая учитывает полученное распределение размеров пачек.

Рассмотрим возможные переходы из состояния $(h,m,n)$ в фронте $F_i$ $i$-ого резервирования в состояния в фронте $F_{i + 1}$ $(i+1)$-ого резервирования, а также прирост среднего числа потерянных пакетов за эти переходы. Опираясь на переходы в случае заранее заданного видеопотока, получим следующую таблицу переходов.

\begin{tabular}{l l l l}
1)	&$(h + \Tres, 0, n)$, 			&$\gamma = 1$, 	&при $m = 0$, $h < -\Tres$; \\
2)  &$(h + \Tres, i, n+1)$, 			&$\gamma = p_i$, &при $m = 0$, $-\Tres \leqslant h < 0$; \\
3)  &$(h + \Tres - \Tin, p_i, n+1)$, 	&$\gamma = (1 - q) p_i$, &при $m = 1$, $\Tin - \Tres \leqslant h \leqslant d$; \\
4)  &$(h + \Tres - \Tin, 0, n)$, 	&$\gamma = 1 - q$, &при $m = 1$, $0 \leqslant h < \Tin - \Tres$; \\
5)  &$(h + \Tres - \Tin, ш, n + 1)$, 	&$\gamma = q p_i$, &при $m = 1$, $d - \Tres < h \leqslant d$; \\
6)  &$(h + \Tres, m, n)$, 			&$\gamma = q$, &при $m > 0$, $0 \leqslant h \leqslant d - \Tres$; \\
7)  &$(h + \Tres - \Tin, i, n+1)$,	&$\gamma = p_i$, &при $m > 1$, $d - \Tres < h \leqslant d$; \\
8)  &$(h + \Tres, m - 1, n)$, &$\gamma = 1 - q$, &при $m > 1$, $0 \leqslant h \leqslant d - \Tres$;
\end{tabular} \newline
где $i\in\{1,\ldots, MAXSIZE\}$ и $\gamma$ --- вероятность перехода.

После описания всевозможных переходов из фронта $F_i$ в фронт $F_{i+1}$ необходимо отнормировать все вероятности (в следующий фронт цепь попадает с вероятностью 1).

\begin{algorithm}
\caption{Алгоритм динамического выбора периода резервирования}
\begin{algorithmic}[1]
\State{$i = 0$}
\State{$\Tres = default value$}
\State{$Distribution$}
\While{i < k}
	\State{Воспроизводим фрагмент № $i$ с заданным $\Tres$}
	\State{Получаем конечное состояние $(h,m,n)$}
	\State{Обновляем $Distribution$}
	\State{Загружаем $(h,m,n)$ и $Distribution$ в аналитическую модель, получив период резервирования для следующего окна $\Tres_{next}$}
	\State{$\Tres := \Tres_{next}$}
\EndWhile
\end{algorithmic}
\end{algorithm}

























