\chapter{Математическая модель одиночной передачи пакетов}
\label{chapter:model_noBlockAck}

В работе было рассмотрено несколько математических моделей...

\section{Аналитическая модель передачи}
\label{analyticalModel1}
\subsection{Цепь Маркова}

Представим $\Tin/\Tres$ в виде несократимой дроби $\tin/\tres$, где $\tin, \tres\in\mathbb{N}$. Назовем слотом интервал времени длины $$\tau = \frac{\Tin}{\tin} \equiv \frac{\Tres}{\tres}.$$ Разобьем непрерывную временную шкалу на слоты таким образом, чтобы начало каждого MCCAOP совпадало с началом некоторого слота -- см.~рис~\ref{fig:time}.

Процесс передачи пакетов с помощью механизма MCCA может быть описан цепью Маркова с дискретным временем, единица которого равна периоду резервирований.

\begin{figure}[t]
\centering{\includegraphics[width=0.7\linewidth]{pics/time1}}
	\caption{\label{fig:time}Слотированное время}
\end{figure}


В каждый момент времени $t$ состояние системы будем описывать парой целых чисел $(h(t), m(t), n(t))$. Если $h(t) \geqslant 0$, то очередь не пуста, и $h(t)$ равно числу полных слотов, которые головная (самая старшая) пачка пакетов провела в очереди, $m(t)$ равно числу пакетов в этой пачке, а $n(t)$ равно начальному числу пакетов в головной пачке. Если $h(t) < 0$, то очередь пуста, и $|h(t)|$ равно времени до прибытия новой пачки пакетов, выраженному в слотах и округленному вниз; при этом $m=0$. Таким образом, состояние системы в каждый моменты времени $t$ характеризуется текущим числом пакетов в головной пачке, начальным числом пакетов  в головной пачке и временем, в течение которого пакеты этой пачки ожидают передачи. Используемые обозначения состояния системы позволяют определить количество пачек пакетов в очереди:  $\lfloor h(t)/\tin \rfloor + 1$, но не позволяют определить длины всех пачек, кроме головной. После того, как все пакеты головной пачки были переданы или отброшены, определяется число пакетов в пачке, ставшей головной.
  
Минимальное значение $h(t)$ равно $\tres-\tin$. Оно достигается в момент времени $t$, если пачка из одного пакета пребывает в пустую очередь непосредственно перед моментом $t - 1$, и единственный пакет пачки успешно передается с первой попытки.   

Теперь найдем максимальное возможное значение $h(t)$. Для этого обозначим через $\xi$ время между поступлением пачки пакетов в очередь и началом следующего слота, $0 \leqslant \xi < \tau$. В силу того, что время $\Tin$ равно целому числу $\tin$ слотов, значение величины $\xi$ одинаково для всех пачек. Таким образом, к моменту времени $t$ время ожидания в очереди пакетов головной пачки равно $h(t) \cdot \tau + \xi$. Чтобы эта пачка не была отброшена в момент $t$, ее время ожидания не должно превышать значения $D$. Следовательно, $h(t) \leqslant d = \lfloor \frac{D-\xi}{\tau} \rfloor$. 

Благодаря тому, что числа $\tin$ и $\tres$ взаимно просты, цепь Маркова обладает свойством эргодичности. Таким образом, может быть найдено стационарное распределение вероятностей цепи Маркова. 

Выясним, в какие состояния и с какой вероятностью может перейти система из состояния $(h, m, n)$ за один шаг. Для этого отдельно рассмотрим несколько случаев.

\paragraph{1.} Пусть $h(t) < 0$, т.~е. в момент времени $t$ очередь пуста. 

\begin{itemize}
\item Если $h + \tres \ge 0$, то к моменту времени $t + 1$ в очередь поступит очередная пачка пакетов. Размер пачки является случайной величиной, подчиняющейся распределению условных вероятностей $p_{ji}$, где $j$ и $i$ --- размеры соседних пачек,  поэтому с вероятностью $p_{ni}$ система окажется в состоянии $(h + \tres, i, i)$. 

\item Если же $h + \tres < 0$, то к моменту времени $t + 1$ в очередь  не поступит ни одного пакета, т.~е. в этом случае система с вероятностью $1$ перейдет в состояние $(h + \tres, 0, n)$. 
\end{itemize}


\paragraph{2.} Пусть $h(t) \ge 0$ и $m(t) = 1$, т.~е. в головной пачке находится единственный пакет.

С вероятностью $1 - q$ происходит успешная передача пакета.

\begin{itemize}
\item Если при этом $h - \tin + \tres < 0$, то к моменту времени $t + 1$ на станцию не поступит ни одной пачки данных и система окажется в состоянии $(h - \tin + \tres, 0, n)$ с указанной выше вероятностью $1 - q$.
\item Если же $h - \tin + \tres \geqslant 0$, то в момент времени $t + 1$ очередь будет не пуста, и состояние системы будет определяться числом пакетов $i$ в головной пачке. Поскольку вероятность успешной попытки передачи пакета равна  $1 - q$, а пачка имеет размер $i$ с вероятностью $p_{ni}$, то система перейдет в состояние $(h - \tin + \tres, i, i)$, $i \in \{1, \dots, M\}$, с вероятностью $(1 - q)p_{ni}$.
\end{itemize} 

С вероятностью $q$ попытка передачи пакета оказывается неудачной. 

\begin{itemize}
\item Если $h + \tres > d$, то к моменту времени $t + 1$ время ожидания данного пакета  превысит допустимое значение $D$, и этот пакет будет отброшен. Cистема c вероятностью $qp_{ni}$ перейдет в одно из состояний $(h - \tin + \tres, i, i)$, $i \in \{1, \dots, M\}$.
\item Если $h + \tres \le d$, то к моменту времени $t + 1$ пакет не устареет, и будет предпринята следующая попытка передачи. Следовательно, система перейдет в состояние $(h + \tres, 1, n)$ с вероятностью $q$, а в состояние $(h + \tres, 0, n)$ с вероятностью $1 - q$.
\end{itemize} 

\paragraph{3.} Наконец, рассмотрим случай, когда $h(t) \geqslant 0$, $m(t) > 1$. 

С вероятностью $1 - q$ попытка передачи пакета была успешной. В этом случае в обслуживаемой пачке остается $m - 1$ пакет. 

\begin{itemize}
\item Если $h + \tres > d$, то к моменту времени $t + 1$ время ожидания этих пакетов превысит допустимое значение $D$, и все пакеты пачки будут отброшены. Система с вероятностью $(1-q)p_{ni}$ перейдет в одно из состояний $(h - \tin + \tres, i, i)$, $i \in \{1, \dots, M\}$. 
\item Если $h + \tres \le d$, то система с вероятностью $1-q$ перейдет в состояние $(h + \tres, m - 1, n)$.
\end{itemize} 

Аналогично, в случае, когда попытка передачи не была успешной, система перейдет в состояние $(h - \tin + \tres, i, i)$ при  $h + \tres > d$ с вероятностью $q p_{ni}$ и в $(h + \tres, m, n)$ --- при $h + \tres \le d$ с вероятностью $q$.

\paragraph{Подводя итог,} получаем, что при выполнении соответствующих условий система  переходит из состояния $(h, m)$ в одно из следующих состояний:

\begin{tabular}{l l l l}
1)	&$(h + \tres, 0, n)$, 			&$\gamma = 1$, 	&при $m = 0$, $h < -\tres$; \\
2)  &$(h + \tres, i, i)$, 			&$\gamma = p_{ni}$, &при $m = 0$, $-\tres \leqslant h < 0$; \\
3)  &$(h + \tres - \tin, i, i)$, 	&$\gamma = (1 - q)p_{ni}$, &при $m = 1$, $\tin - \tres \leqslant h \leqslant d$; \\
4)  &$(h + \tres - \tin, 0, n)$, 	&$\gamma = 1 - q$, &при $m = 1$, $0 \leqslant h < \tin - \tres$; \\
5)  &$(h + \tres - \tin, i, i)$, 	&$\gamma = q p_{ni}$, &при $m = 1$, $d - \tres < h \leqslant d$; \\
6)  &$(h + \tres, m, n)$, 			&$\gamma = q$, &при $m > 0$, $0 \leqslant h \leqslant d - \tres$; \\
7)  &$(h + \tres - \tin, i, i)$,	&$\gamma = p_{ni}$, &при $m > 1$, $d - \tres < h \leqslant d$; \\
8)  &$(h + \tres, m - 1, n)$, &$\gamma = 1 - q$, &при $m > 1$, $0 \leqslant h \leqslant d - \tres$;
\end{tabular} \newline
где $i\in\{1,\dots,M\}$ и $\gamma$ --- вероятность перехода. 

Уделим особое внимание переходу под номером~$7$. Он может осуществиться двумя способами. 

Первый способ заключается в том, что сначала с вероятностью $1 - q$ происходит успешная передача пакета, а потом с вероятностью $p_i$ переход в состояние $(h + \tres - \tin, i, i)$. Таким образом, вероятность перехода первым способом равна $(1 - q) p_{ni}$.

Второй способ заключается в том, что сначала с вероятностью $q$ возникает ошибка при передаче пакета, а затем с вероятностью $p_{ni}$ переход в состояние $(h + \tres - \tin, i, i)$. Получаем, что  вероятность перехода вторым способом равна $q p_{ni}$, а суммарная вероятность перехода $7$ равна $p_{ni}$. 

Однако между этими двумя способами реализации перехода есть существенное различие: при~первом способе отбрасывается $m - 1$ пакет, а при втором~--- $m$ пакетов. Данное обстоятельство окажется существенным при~подсчете значения $PLR$.  

\subsection{Определение PLR}
Упорядочив состояния в лексикографическом порядке и построив матрицу $P$ переходных вероятностей, найдем стационарные вероятности $\pi_{(h,m,n)}$ состояний $(h,m,n)$.

Зная станционарные вероятности $\pi_{(h,m, n)}$, найдем долю $PLR$ потерянных пакетов. Т.~к. пакеты отбрасываются только при превышении порога времени $D$ ожидания в очереди, потери пакетов могут происходить только при следующих переходах:
\begin{itemize}
\item[1)]переход из состояния $(h, m, n)$ в состояние \mbox{$(h - \tin + \tres, i, i)$} с вероятностью $q p_{ni}$ при условиях $m = 1$ и $d - \tres < h \leqslant d$ (отбрасывается $1$ пакет);
\item[2)]переход из состояния $(h, m, n)$ в состояние \mbox{$(h - \tin + \tres, i, i)$} с вероятностью $(1 - q)p_{ni}$ при условиях $m > 1$ и $d - \tres < h \leqslant d$ (отбрасывается $m - 1$ пакет); 
\item[3)]переход из состояния $(h, m, n)$ в состояние \mbox{$(h - \tin + \tres, i)$} с вероятностью $q p_{ni}$ при условиях $m > 1$ и $d - \tres < h \leqslant d$ (отбрасывается $m$ пакетов);
\end{itemize}

Таким образом, среднее число пакетов, отбрасываемых за один шаг модельного времени ($\Tres$), равно
$$
	\Idis = \sum\limits_{(h,m, n)\colon h > d - \tres} \left((m - 1)(1 - q)\pi_{(h,m,n)} + mq\pi_{(h,m,n)}\right) = 
	\sum\limits_{(h,m,n)\colon h > d - \tres} (m - 1 + q)\pi_{(h,m,n)}.
$$

Значение $PLR$ равно отношению среднего числа $\Idis$ пакетов, отброшенных за один шаг модельного времени, к среднему числу $\Iin$ пакетов, поступивших в очередь за то же время:
$$
PLR = \frac{\Idis}{\Iin} = \frac{1}{\frac{\Tres}{\Tin} \sum\limits_{i} i\cdot p_i} \cdot \left(\sum\limits_{(h,m,n)\colon h > d - \tres} (m - 1 + q)\pi_{(h,m,n)} \right).
$$

\subsection{Динамическая аналитическая модель} %????
\label{DynamicModel}
Рассмотрим передачу пакетов следующим образом. Будем считать полностью известным входной поток $B$, т.~е. наперёд известен размер потока $l$ в пачках и размер каждой из них $B_i, i=\bar{1, l}$ в пакетах. Пусть состояние системы будет описываться тройкой параметров $(h,m,n)$, где $h$ и $m$ обозначают то же, что и в ранее описанной модели, а $n$ равно номеру старшей пачки в потоке.

Назовём фронтом состояний системы в момент времени $t$ множество всех возможных состояний $(h(t), m(t), n(t))$, в которые система может попасть с ненулевой вероятностью. Будем рассматривать систему в моменты наступления периодов резервирования, в каждый из которых она имеет свой фронт $F_i$, где $i$ --- номер зарезервированного интервала в процессе передачи. Начнём рассматривать систему в момент наступления первого периода резервирования. Таким образом, в начальный момент времени система находится в состоянии $(\xi, B_1, 1)$. Также каждому состоянию будет соответствовать вероятность попадания в него и среднее число отброшенных пакетов по его достижении.

Рассмотрим возможные переходы из состояния $(h,m,n)$ в фронте $F_i$ $i$-ого резервирования в состояния в фронте $F_{i + 1}$ $(i+1)$-ого резервирования, а также прирост среднего числа потерянных пакетов за эти переходы.

\paragraph{1.} $h < 0$, т.~е. очередь пуста.

\begin{itemize}
\item Если $h + \tres \ge 0$, то к моменту времени $t + 1$ в очередь поступит очередная пачка пакетов. Размер пачки заранее определён, поэтому с вероятностью $1$ система окажется в состоянии $(h + \tres, B_{i+1}, i+1)$.

\item Если же $h + \tres < 0$, то к моменту времени $t + 1$ в очередь не поступит ни одного пакета, т.~е. в этом случае система с вероятностью $1$ перейдет в состояние $(h + \tres, 0, i)$. 
\end{itemize}

В обоих случаях среднее число потерянных пакетов за переход равно нулю.
%%%W%%%%%%%%%%%%
\paragraph{2.} Пусть $h(t) \ge 0$ и $m(t) = 1$, т.~е. в головной пачке находится единственный пакет.

\begin{itemize}
\item Если при этом $h - \tin + \tres < 0$, то к моменту времени $t + 1$ на станцию не поступит ни одной пачки данных и система окажется в состоянии $(h - \tin + \tres, 0, n)$ с вероятностью $1$.
\item Если же $h - \tin + \tres \geqslant 0$, то в момент времени $t + 1$ очередь будет не пуста, и состояние системы будет определяться числом пакетов $i$ в головной пачке, тогда система перейдет в состояние $(h - \tin + \tres, B_{n+1}, n+1)$ с вероятностью $1$.
\end{itemize}

В обоих случаях среднее число потерянных пакетов за переход равно $q$.

\paragraph{3.} Наконец, рассмотрим случай, когда $h(t) \geqslant 0$, $m(t) > 1$. 

\begin{itemize}
\item Если $h + \tres > d$, то к моменту времени $t + 1$ время ожидания этих пакетов превысит допустимое значение $D$, и все пакеты пачки будут отброшены. Система с вероятностью $1$ перейдет в состояние $(h - \tin + \tres, B_{n+1}, n+1)$. Здесь среднее число потерянных пакетов за переход равно $m - 1 + q$.
\item Если $h + \tres \le d$, то система с вероятностью $1 - q$ перейдет в состояние $(h + \tres, m - 1, n)$, а с вероятностью $q$ перейдёт в состояние $(h + \tres, m, n)$. Здесь в обоих переходах среднее число потерянных пакетов равно нулю.
\end{itemize} 





Видно, что они не отличаются от переходов предыдущей модели, за исключением того, что поскольку в рамках данной модели полностью известен поток, размер новой головной пачки задается однозначно







\subsection{Выходной поток}
\label{outputStream1}
Для моделирования передачи потока на последующих шагах нужно найти распределение $\{r_k\}$ интервалов времени между двумя последовательными пакетами выходного потока.
Т.~к. пакеты передаются только во время резервирований, то интервалы времени $\Tout_k$ между пакетами кратны периоду резервирований $\Tres$, т.~е. $\forall k \in \mathbb{N} \longrightarrow \Tout_k = k \cdot \Tres$.

Пусть в момент $t-1$ модельного времени произошла успешная передача пакета, а в момент $t$ значение $h(t)$ равно $j$. Найдем $\theta_l^j$ вероятность того, что следующая успешная передача произойдет в момент времени $t + l$. 

При $l = 0$ значение $\theta_l^j$ равно вероятности успешной передачи пакета в ближайшее резервирование, т.~е.:

\begin{equation}
	\label{theta1}
	\theta_0^j = \begin{cases}
	1-q, &j \geqslant 0; \\
	0, &j < 0.
	\end{cases}
\end{equation}

Для $l > 0$ найдем значение $\theta_l^j$ по индукции, заметив, что событие <<следующая успешная передача состоится через $l$ MCCAOP>> эквивалентно следующим условиям:
\begin{itemize}
	\item[1)] в текущем MCCAOP (т.~е. в момент $t$), не будет успешной передачи пакета;
	\item[2)] относительно момента времени $t + 1$ имеет место событие <<следующая успешная передача состоится через $l-1$ MCCAOP>>.
\end{itemize}


Учитывая это, выразим вероятности $\{\theta_l^j\}$ через вероятности $\{\theta_{l-1}^j\}$:
\begin{itemize}
	\item Если $h(t) = j < 0$, то очередь пуста, и с вероятностью 1 в текущем MCCAOP не будет передачи пакета. Следовательно, $h(t+1)=j+\tres$ и $\theta_{l}^j = \theta_{l-1}^{j+\tres}$.
	\item Если $0 \leqslant j \leqslant d - \tres$, то очередь не пуста, и с вероятностью $q$ передача пакета в текущем MCCAOP будет неуспешной. При этом к моменту времени $t + 1$ головная пачка пакетов не устареет. Следовательно, $h(t + 1) = j + \tres$ и $\theta_{l}^j = 
q \cdot\theta_{l-1}^{j+\tres}$.
	\item Если $j > d - \tres$, то с вероятностью $q$ передача пакета в текущем MCCAOP будет неуспешной. При этом к моменту времени $t + 1$ головная пачка пакетов устареет и будет отброшена. Следовательно, $h(t + 1) = j + \tres - \tin$ и $\theta_{l}^j  = q \cdot \theta_{l-1}^{j+\tres-\tin}$.
\end{itemize}

В результате, значение $\theta_{l}^j$ при $l > 0$ находится следующим образом
\begin{equation}
	\label{theta2}
	\theta_{l}^j = \begin{cases}
		q\cdot \theta_{l-1}^{j+\tres}, & 0 \leqslant j \leqslant d - \tres; \\
		q\cdot \theta_{l-1}^{j+\tres-\tin}, &j > d - \tres; \\
		\theta_{l-1}^{j+\tres}, &j < 0. 
	\end{cases}
\end{equation} 
Объединяя \eqref{theta1} и \eqref{theta2}, можем вычислить распределение $\{\theta_l^j\}$. 

Для дальнейшего использования построенной модели необходимо, чтобы распределение $\{\theta_l^j\}$ было ограниченным: $\exists l_{max} \in \mathbb{N} \colon \forall l > l_{max} \longrightarrow \theta_l^j = 0$. Поэтому построим близкое к $\{\theta_l^j\}$ ограниченное распределение $\{\hat{\theta}_l^j\}$:
\begin{equation}
	\hat{\theta}_l^j = \begin{cases}
		\theta_l^j, &l < l_{max};	\\
		\sum\limits_{i = l_{max}}^{\infty} \theta_i^j, &l = l_{max}; 	\\
		0,	&l > l_{max}. 
	\end{cases}
\end{equation} 

Пусть $\rho_j$ вероятность того, что в начале MCCAOP, следующего за успешной передачей пакета, система находится в состоянии $h(t) = j$. Тогда для ограниченного распределения $\{\hat{\theta}_l^j\}$ общая вероятность $r_k$ того, что интервал времени между двумя последовательными успешными передачами равен $\Tout_k$ вычисляется как:
\begin{equation}
	\label{interTimeProb1}
	r_k = \sum\limits_{j} \rho_j \cdot \hat{\theta}_{k-1}^j.	
\end{equation}

Вычислим значения $\rho_j$. Для нахождения $\rho_j$ заметим, что после успешной передачи в момент времени $t$ переход в состояние с $h(t+1)=j$ возможен только из следующих состояний $(h(t),m(t))$:
\begin{itemize}
	\item[1)] $h(t) = j - \tres$, $m(t) > 1$, если $0 \leqslant j - \tres \leqslant d$;
	\item[2)] $h(t) = j - \tres + \tin$, $m(t) = 1$, если  $0 \leqslant j + \tin - \tres \leqslant d - \tres$;
	\item[3)] $h(t) = j - \tres + \tin$, $m(t) \geqslant 1$, если $d - \tres < j + \tin - \tres \leqslant d$.
\end{itemize}
Получаем:
\begin{multline}
	\label{conditionProb1}
	\rho_j = \frac{1}{\sum\limits_{(h,m)\colon h \geqslant 0} \pi_{(h,m)}} \cdot 
	\biggl( {\sum\limits_{\substack{(h,m)\colon \\ h = j - \tres, m > 1}} \pi_{(h,m)}} \cdot
	1\{\tres \leqslant j \leqslant d\} + \\
	+ {\sum\limits_{\substack{(h,m)\colon \\ h = j - \tres + \tin, m = 1}} \pi_{(h,m)}} \cdot 
	1\{\tres - \tin \leqslant j \leqslant d - \tin\}	 +	\\
	+  {\sum\limits_{\substack{(h,m)\colon \\ h = j - \tres + \tin, m \geqslant 1}} \pi_{(h,m)}} \cdot
	1\{d - \tin < j \leqslant d - \tin + \tres\} \biggl), 
\end{multline}
где $1\{\mathcal{A}\}$ --- индикатор:
\begin{equation}
	\notag
	1\{\mathcal{A}\} = \begin{cases}
		1, $		если условие $\mathcal{A}$ выполнено;$\\
		0, $		если условие $\mathcal{A}$ не выполнено.$
	\end{cases}
\end{equation}
Подстановка \eqref{conditionProb1} в  \eqref{interTimeProb1} дает искомый результат.



\section{Аналитическая модель передачи на остальных шагах}
\label{analyticalModel2}
\subsection{Цепь Маркова}
Входной поток для всех шагов, кроме первого, представляет собой ординарный поток пакетов, интервалы времени между которыми являются случайными величинами. Аналитическая модель для случая такого входного потока уже была построена в работе \cite{shvets2011stream}. Для создания цельной картины приведем ее краткое описание. 

Рассмотрим передачу пакетов шаге, отличном от первого. Далее будем опускать индекс $j > 1$, являющийся номером шага. 

При построении модели будем считать, что входной поток представлен значениями $\{\Tin_i\}$ интервалов времени между поступлениями пакетов и их распределением вероятностей $\{p_i\}$, причем распределение $\{\Tin_i\}$ является ограниченным, т.~е. $\exists M \in \mathbb{N} \colon p_M \ne 0, \forall i > M \longrightarrow p_i = 0$;

Длительность $\tau$ слота определятся, как наибольший общий делитель чисел $\Tres, \Tin_1, \dots, \Tin_M$. При этом введем следующие обозначения:
\begin{gather}
	\tres = \Tres/\tau, \tres\in\mathbb{N}, \notag \\
	\tin_i = \Tin_i/\tau, \tin_i \in \mathbb{N}. \notag
\end{gather}

Процесс передачи ординарного непериодического потока, как и в уже построенной модели, описывается с помощью цепи Маркова с дискретным временем, только состояние цепи Маркова в некоторый момент времени $t$ описывается единственным числом $h(t)$, которое лежит в диапазоне от $\tres - \tin_M$ до $d$, где $d = \lfloor \frac{D-\xi}{\tau} \rfloor$.

При выполнении соответствующих условий система переходит из состояния $h$ в одно из следующих состояний:

\begin{tabular}{l l l l}
1) &$h + \tres$, &$\gamma = 1$, &если $h < 0$; \\
2) &$h + \tres$, &$\gamma = q$, &если $0 \leqslant h \leqslant d - \tres$; \\
3) &$h+\tres-\tin_i$, &$\gamma = (1-q) p_i$, &если $0 \leqslant h \leqslant d - \tres$; \\
4) &$h+\tres-\tin_i$, &$\gamma = p_i$, &если $d - \tres < h \leqslant d - \tres + \tin_1$; \\
5) &$h+\tres-\tin_1 - \tin_i$, &$\gamma = p_1 p_i$, &если $d - \tres + \tin_1 < h \leqslant d$; \\
6) &$h+\tres-\tin_i$, &$\gamma = p_i$, &если $d - \tres + \tin_1 < h \leqslant d$, $i\ne 1$. 
\end{tabular} \newline
где $\gamma$ -- вероятность перехода, $i\in\{1,\dots,M\}$. 
Данные переходы получены при следующих условиях: $\tres \leqslant d$, $\tres < \tin_2$, \mbox{$\tres \leqslant 2\tin_1$}.

\subsection{Определение PLR}
Определив стационарные вероятности $\pi_h$ состояний $h$, вычислим среднее число пакетов, отбрасываемых за один шаг модельного времени по формуле:
\begin{gather}
	\notag
	\Idis = \sum\limits_{h\colon d - \tres < h \le d - \tres + \tin_1} \pi_h\cdot q +
	\sum\limits_{h\colon d - \tres + \tin_1 < h \le d} \pi_h\cdot 
	\left(q(1-p_1)+(1-q)p_1+2qp_1\right) =\\
	= \sum\limits_{h\colon d - \tres < h \le d - \tres + \tin_1} \pi_h\cdot q +
	\sum\limits_{h\colon d - \tres + \tin_1 < h \le d} \pi_h\cdot 
	\left(q+p_1\right)
\end{gather}

Значение $PLR$ равно отношению среднего числа $\Idis$ отброшенных пакетов к среднему числу $\Iin$ пакетов, поступивших в очередь:
$$
	PLR = \frac{\Idis}{\Iin} = \frac{1}{\frac{\Tres}{\sum\limits_{i} i\cdot \Tin_i}} \cdot \left( \sum\limits_{h\colon d - \tres < h \le d - \tres + \tin_1} \pi_h\cdot q +
	\sum\limits_{h\colon d - \tres + \tin_1 < h \le d} \pi_h\cdot 
	\left(q+p_1\right) \right)
$$
\subsection{Выходной поток}
Найдем распределение вероятностей $\{r_k\}$ интервалов времени $\{\Tout_k\}$ между двумя последовательными пакетами выходного потока точно так же, как и в разделе \ref{outputStream1}.
 
Пусть в момент времени $t-1$ произошла успешная передача пакета, а в момент $t$ значение $h(t)$ равно $j$. Как и ранее, $\theta_l^j$ есть вероятность того, что следующая успешная передача произойдет в момент времени $t + l$. Соответственно,
\begin{equation}
	\label{theta3}
	\theta_0^j = \begin{cases}
	1-q, &j \geqslant 0; \\
	0, &j < 0,
	\end{cases}
\end{equation}
а значения $\{\theta_l^j\}$ для $l>0$ определяются по индукции:
\begin{equation}
	\theta_{l}^j = \begin{cases}
		\theta_{l-1}^{j+\tres}, &j < 0;	\\
		q\theta_{l-1}^{j+\tres}, &0 \leqslant j \leqslant d - \tres; \\
		q \sum\limits_{i = 1}^M p_i\theta_{l-1}^{j+\tres-\tin_i}, 
			& d - \tres < j \leqslant d - \tres + \tin_1; \\
		q \sum\limits_{i=2}^M p_i  \theta_{l-1}^{j+\tres-\tin_i} + 
			q p_1 \sum\limits_{i=1}^M p_i  \theta_{l-1}^{j+\tres-\tin_1-\tin_i},
			&d - \tres + \tin_1 < j \leqslant d.
	\end{cases}
	\label{theta4}
\end{equation} 


Как и ранее для дальнейшего использования построенной модели необходимо, чтобы распределение $\{\theta_l^j\}$ было ограниченным: $\exists l_{max} \in \mathbb{N} \colon \forall l > l_{max} \longrightarrow \theta_l^j = 0$. Поэтому построим близкое к $\{\theta_l^j\}$ ограниченное распределение $\{\hat{\theta}_l^j\}$:
%Введем ограниченное распределение $\{\hat{\theta}_l^j\}$:
\begin{equation}
	\hat{\theta}_l^j = \begin{cases}
		\theta_l^j, &l < l_{max};	\\
		\sum\limits_{i = l_{max}}^{\infty} \theta_i^j, &l = l_{max}; 	\\
		0,	&l > l_{max}. 
	\end{cases}
\end{equation} 

Пусть $\rho_j$ вероятность того, что в начале MCCAOP следующего за успешной передачей пакета, система находится в состоянии $h(t) = j$. Тогда вероятность $r_k$ того, что интервал времени между двумя последовательными успешными передачами равен $\Tout_k$ вычисляется как:
\begin{equation}
	\label{interTimeProb}
	r_k = \sum\limits_{j} \rho_j \cdot \hat{\theta}_{k-1}^j.	
\end{equation}

Для нахождения $\rho_j$ заметим, что после успешной передачи в момент времени $t$ переход в состояние с $h(t+1)=j$ возможен только из следующих состояний $h(t)$:
\begin{itemize}
	\item[1)] $h(t) = j - \tres + \tin_i$, если $0 \leqslant j - \tres + \tin_i \leqslant d$;
	\item[2)] $h(t) = j - \tres + \tin_1 + \tin_i$, если  
				$d - \tres + \tin_1 < j - \tres + \tin_1 + \tin_i \leqslant d$.
\end{itemize}
Откуда получаем:
\begin{multline}
\label{conditionProb}
	\rho_j = \frac{1}{\sum\limits_{h = 0}^d\pi_h}\cdot
	\left(\sum\limits_{i\colon\tres-\tin_i \leqslant j \leqslant d+\tres-\tin_i}\pi_{j-\tres+\tin_i}\cdot 
		p_i + \right. \\
	 +\left.\sum\limits_{i\colon d - \tin_i < j \leqslant d+\tres-\tin_1-\tin_i}\pi_{j-\tres+\tin_1+\tin_i}\cdot 
	 p_1 p_i\right). 
\end{multline}
Подстановка \eqref{conditionProb} в  \eqref{interTimeProb} дает искомый результат.
